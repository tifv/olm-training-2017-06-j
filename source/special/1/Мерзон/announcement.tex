% $date: 2--5 июня 2017 г.

\worksheet*{Целые точки в многоугольниках и многогранниках}

% $authors:
% - Григорий Александрович Мерзон

% $style[-announcement]:
% - .[announcement]

Как найти площадь многоугольника на~клетчатой бумаге?
Оценку <<площадь примерно равна числу занятых клеток>> можно превратить
в~точную формулу (<<формула Пика>>).
Мы поговорим о~том, как можно эту формулу понимать и~как ее правильно обобщать
(начало <<теории Эрхарта>>).

Возникающая теория оказывается применима в~разных задачах алгебры
и~комбинаторики, в~которых никаких геометрических фигур, на~первый взгляд,
не~видно.
Об~этом мы тоже немного поговорим.

Предварительное знание формулы Пика не~требуется (и~не~мешает).

