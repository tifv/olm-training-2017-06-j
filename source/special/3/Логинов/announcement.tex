% $date: 10--13 июня 2017 г.

\worksheet*{Математика узлов}

% $authors:
% - Константин Валерьевич Логинов

% $style[-announcement]:
% - .[announcement]

Мы будем исследовать узлы.
Это математические объекты, с~которыми можно встретиться и~в~повседневной
жизни, например, когда на~ботинках развязываются шнурки или у~наушников
спутываются провода.

Мы изучим, как доказать, что одни узлы можно развязать, а~другие нет,
посмотрим, как узел можно раскрасить в~три цвета (и~зачем это делать), а~также
свяжем с~узлами многочлены, потому что и~в~этой науке без алгебры никак нельзя.
Большую часть времени мы будем рисовать картинки узлов и~преобразовывать их,
но~любителям работать с~буквенными выражениями  тоже скучно не~будет.
