% $date: 10--13 июня 2017 г.

\worksheet*{Экстремальные графы}

% $authors:
% - Владимир Викторович Трушков

% $style[-announcement]:
% - .[announcement]

В~курсе мы изучим вопрос, каким может быть максимальное количество ребер графа
на~$n$~вершинах, если в~нем запрещен какой-то подграф.
Например, нет трех попарно соединенных вершин.
В~некоторых задачах удается получить точный ответ, т.\,е. и~доказать оценку,
и~привести пример.
А~в~других, например, в~задаче про запрет квадрата (т.\,е. нет циклов
на~4~вершинах) точного ответа нет, но~от~этого она не~становится менее
интересной.
Полезность этой теории мы проиллюстрируем на~задачах с~олимпиад и~турниров
матбоев.

