% $date: 10--13 июня 2017 г.

\worksheet*{Введение в теорию множеств}

% $authors:
% - Иван Викторович Митрофанов

% $style[-announcement]:
% - .[announcement]

Приглашаются те, кто слышал слова <<счётное множество>>, <<континуум>> и~тому
подобное, но~не~очень глубоко в~этом разбирается.
Или вообще этого не~слышал.

Мы научимся сравнивать бесконечные множества по~размеру, поймём, почему
натуральных чисел столько~же, сколько и~рациональных, а~действительных
чисел~--- больше (и~как эти действительные числа строго определить), почему
точек в~любом отрезке столько~же, сколько точек во~всей Вселенной, а~точек
во~Вселенной меньше, чем число способов покрасить точки отрезка в~два цвета.
А~ещё мы докажем пару именных теорем и~научимся решать бесконечные
конструктивы.
