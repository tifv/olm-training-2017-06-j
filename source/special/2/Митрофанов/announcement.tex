% $date: 6--9 июня 2017 г.

\worksheet*{Конечные автоматы и синхронизирующие раскраски}

% $authors:
% - Иван Викторович Митрофанов

% $style[-announcement]:
% - .[announcement]

Представьте ориентированный граф, ребра которого покрашены в~синий, красный
и~зеленый цвета.
Пусть из~каждой вершины выходит ровно по~одному ребру каждого цвета
и~в~какой-то вершине сидит умный червячок,
различающий цвета, умеющий ползать по~ребрам и~понимающий простые команды.
Тогда, получив последовательность команд <<зеленый, зеленый, красный, синий>>,
он проползет четыре ребра и~попадет в~новую вершину.
Можно~ли составить такую последовательность указаний (может быть, длинную),
что, выполнив ее, червячок оказывается в~одной и~той~же вершине графа
\emph{вне зависимости} от~того, из~какой вершины начинал?

Такой граф с~цветными ребрами называется графом состояний конечного автомата,
эта задача~--- задачей синхронизации автомата, а~последовательность команд,
сгоняющая всевозможных червячков в~одну вершину~--- синхронизирующим словом.

Никаких предварительных знаний не~требуется, но~простым курс не~будет, так как
во~второй его половине мы разберем доказательство гипотезы
(а~с~2007 года~--- теоремы) о~раскраске дорог (road coloring problem).
Если нам дан орграф (не~раскрашенный), при каком условии его ребра можно
раскрасить так, чтобы для полученного автомата существовало синхронизирующее
слово?
Были найдены два очевидных необходимых условия: сильная связность
и~непериодичность (что это такое, я расскажу на~занятии).
Спустя 37~лет А.\,Трахтман доказал, что эти~же условия являются и~достаточными.

