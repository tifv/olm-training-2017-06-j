% $date: 6--9 июня 2017 г.

\worksheet*{Теория игр}

% $authors:
% - Владимир Алексеевич Брагин

% $style[-announcement]:
% - .[announcement]

На~этом спецкурсе рассмотрим несколько сюжетов, связанных с~математическими
играми.

Сначала изучим беспристрастные игры, в~которых в~любой ситуации оба игрока
могут сделать одни и~те~же ходы.
Очень много игр из~олимпиадных задач именно такие.
Интересно, что все они в~каком-то смысле похожи на~игру \textbf{<<NIM>>}, это
мы и~установим.
Тем, кто знает функцию Шпрага-Гранди будет не~очень ново то, что
рассматривается в~этом блоке, тем не~менее будет возможность решить несколько
задач на~эту тему.

Если успеем, то~посмотрим на~обобщение этой теории для игр, где у~разных
игроков разные ходы.
Оказывается, многие из~них (почти все!?) сводятся к~довольно простой в~описании
игре \textbf{<<Hackenbush>>}.

Последний сюжет~--- в~каком-то смысле заминка, потому что не~будет сложных
доказательств и~задач, требующих долгого счета.
Это уже не~комбинаторная теория игр, а~обыкновенная.
Речь пойдет о~так называемых играх в~нормальной форме.
Когда у~каждого из~игроков есть множество стратегий и~все они одновременно
выбирают стратегию, которой будут придерживаться, а~в~зависимости от~выбранных
стратегий начисляется выигрыш каждого игрока.
Такой подход позволяет рассматривать многие события из~жизни как математическую
игру, поэтому речь зайдет и~о~том, какое отражение те или иные вещи
(например, \emph{дилемма заключенного}) находят в~жизни.

Для зачета надо будет сдать домашнее задание, требующее вдумчиво потраченного
времени.

Будет интересно, но~не~очень просто.
Приходите!

