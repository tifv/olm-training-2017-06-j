% $date: 6--9 июня 2017 г.

\worksheet*{Задачи для записи в~\LaTeX, часть 1}

% $authors:
% - Юлий Васильевич Тихонов

% $build$matter[print]: [[.], [.]]

\emph{Можно брать и другие понравившиеся вам задачи из занятий, но нужно
показать их мне сначала.}

\subsection*{Комбинаторика}

\begin{problems}

\item
Банда из~25 гангстеров устроила перестрелку, в~которой каждый из~них получил
пулю от~ближайшего гангстера
(они сидели в~укрытиях и~не~перемещались, все попарные расстояния между ними
были различны).
Докажите, что какой-то гангстер не~подстрелил никого.

\item
Сколькими способами можно выстроить класс из~$20$~человек в~хоровод, чтобы Петя
был на~одинаковом расстоянии от~Кати и~Маши?

\item
Имеется 20~человек~--- 10~юношей и~10~девушек.
Сколько существует способов составить компанию, в~которой было~бы одинаковое
число юношей и~девушек?

\item
Назовем девятизначное число \emph{хорошим,} если в~нем можно переставить одну
цифру на~другое место и~получить девятизначное число, в~котором цифры идут
строго по~убыванию.
Сколько всего хороших чисел?

\item
На столе лежит куча из $1001$ камня.
Ход состоит в том, что из какой-то кучи, содержащей более одного камня,
выкидывают камень, а затем одну из куч делят на две.
Можно ли через несколько ходов оставить на столе только кучки, состоящие из
трех камней?

\item
Докажите, что за $4$~взвешивания нельзя среди $100$ одинаковых монет найти
одну фальшивую, которая легче настоящих.

\end{problems}

\subsection*{Алгебра}

\begin{problems}

\item
Пусть $a, b, c$~--- натуральные числа, $\text{НОД}(a, b, c) = 1$
и~$\frac{a b}{a - b} = c$.
Докажите, что $a - b$ является точным квадратом.

\item
Пусть $a$, $b$ и~$c$~--- попарно взаимно простые натуральные числа.
Найдите все возможные значения
$\dfrac{(a+b)(b+c)(c+a)}{abc}$, если известно, что это число целое.

\item
Последовательность натуральных чисел $a_{i}$ такова, что
$\text{НОД}(a_{i}, a_{j}) = \text{НОД}(i, j)$ для всех $i \neq j$.
Докажите, что $a_{i} = i$  для всех натуральных $i$.

\item
Докажите, что при $n \neq m$ числа $a^{2^{n}} + 1$ и $a^{2^{m}} + 1$ будут
взаимно простыми.

\item
Решите в натуральных числах уравнение $3 \cdot 2^m + 1 = n^2$.

\item
При каких натуральных~$n$ многочлен $x^{n} + x + 1$ делится на $x^{2} + x + 1$?

\end{problems}

\emph{Геометрические задачи — в следующей серии.}

