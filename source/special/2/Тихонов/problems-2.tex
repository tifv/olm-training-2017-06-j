% $date: 6--9 июня 2017 г.

\worksheet*{Задачи для записи в~\LaTeX, часть 2}

% $authors:
% - Юлий Васильевич Тихонов

% $build$matter[print]: [[.], [.]]

\subsection*{Геометрия}

\begin{problems}

\item
Точка $K$~--- середина гипотенузы $AB$ прямоугольного треугольника $ABC$.
На~катетах $AC$ и~$BC$ выбраны точки
$M$ и~$N$ соответственно так, что угол
$MKN$~--- прямой.
Докажите, что из~отрезков $AM$, $BN$ и~$MN$ можно составить прямоугольный
треугольник.

\item
В треугольнике $ABC$ точка $M$~--- середина $AC$. На стороне $BC$ взяли точку $K$ так, что угол $BMK$ прямой. 
Оказалось, что $BK=AB$. Найдите $MBC$, если $\angle ABC =110^{\circ}$.

\item
Дан треугольник $ABC$. На его сторонах $AB$ и $BC$ построены внешним 
образом квадраты $ABMN$ и $BCPQ$. Докажите, что центры этих квадратов и середины отрезков 
$MQ$ и $AC$ образуют квадрат.

\item
Дан правильный шестиугольник $ABCDEF$. Докажите, что точка $A$ и середины отрезков $BD$ и $EF$ являются вершинами правильного треугольника.

\item
Пусть $AL$ --- биссектриса треугольника $ABC$. Известно, что $\angle ABC = 2\angle ACB+\angle BAC$. Докажите, что $AB + CL = AC$.

\item
Биссектрисы углов треугольника $ABC$ пересекаются в точке $I$. Известно, что $CA+AI = BC$. Найдите отношение углов $BAC$ и $CBA$.

\end{problems}

