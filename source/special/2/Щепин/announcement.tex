% $date: 6--9 июня 2017 г.

\worksheet*{Экспонента и логарифм}

% $authors:
% - Евгений Витальевич Щепин

% $style[-announcement]:
% - .[announcement]

Гиперболический логарифм числа~$x$ определяется как площадь под графиком
гиперболы $y = 1 / x$ на~отрезке от единицы до~$x$.
Экспонента числа~$x$ определяется как сумма бесконечного ряда
\[
    1 + x + \frac{x^{2}}{2} + \frac{x^{3}}{3!} + \frac{x^{4}}{4!} + \ldots +
    \frac{x^{n}}{n!} + \ldots
\]
Основной результат спецкурса~--- теорема о взаимной обратимости
гиперболического логарифма и экспоненты~--- будет строго доказан.
Для сдачи зачета нужно будет решить достаточное количество предложенных задач.
Задач будет много.

