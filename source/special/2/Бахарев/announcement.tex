% $date: 6--9 июня 2017 г.

\worksheet*{Рождественская теорема Ферма}

% $authors:
% - Фёдор Львович Бахарев

% $style[-announcement]:
% - .[announcement]

Какие числа представляются в~виде суммы двух квадратов?
Это очень древняя проблема, она является классической в~теории чисел.
Основная часть решения состоит в~доказательстве возможности представления
простого вида $4 k + 1$ в~виде суммы двух квадратов.
Это утверждение называется <<Рождественская теорема Ферма>>.

Изначально проблема была поставлена французским математиком Альбером Жираром
в~1632\,г.
Он рассмотрел ряд простых чисел и~отмечал в~нем числа, которые представлялись
в~виде суммы двух квадратов.
Конечно~же, простые числа вида $4 k + 3$ у~него представить не~получилось,
но~зато простые числа вида $4 k + 1$ в~пределах сотни все представились.
На~основании этого он выдвинул гипотезу о~том, что простые числа вида $4 k + 1$
представимы в~виде суммы двух квадратов, единственное четное простое
число~2~--- представимо, а~остальные простые~--- нет.
От~разочарования Жирар умер в~том~же 1632 году, так и~не~доказав этот факт.

Доказан~же он был Пьером Ферма лишь восемь лет спустя.
Хотя, как это нередко бывало с~Ферма, предлагается поверить ему на~слово:
с~присущим ему оптимизмом он сообщил о~существовании воистину прелестнейшего
доказательства в~письме Марену Мерсенну от~25 декабря 1640\,г.

Как несложно догадаться, именно дата этого письма и~дала теореме ее праздничное
название.
О~доказательствах этой теоремы и~пойдет речь в~спецкурсе.
По~мнению Г. Г. Харди, эта теорема
<<вполне справедливо считается одной из~наиболее совершенных в~арифметике>>.
Мы~же попробуем подтвердить его слова, доказав эту теорему несколькими
различными способами.

