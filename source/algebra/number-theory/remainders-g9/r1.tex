% $date: 2017-06-02
% $timetable:
%   g9r1:
%     2017-06-02:
%       2:

\worksheet*{Приветственный разнобой по~теории чисел}

% $authors:
% - Ольга Дмитриевна Телешева

\begin{problems}

\item
Дано натуральное число $a$.
Докажите, что при некотором натуральном $n$ у~числа
$1 + a + a^{2} + a^{3} + \ldots + a^{n}$
больше миллиона различных простых делителей.

\item
Докажите, что $(a^{n} - 1, a^{m} - 1) = a^{(n,m)} - 1$.

\item
Докажите, что при $m \neq n$ будут взаимно простыми числа:
\\
\subproblem $a^{2^m} + 1$ и~$a^{2^n} + 1$ для чётных $a$;
\\
\subproblem $a^{2^{m+1}} - a^{2^m} + 1$ и~$a^{2^{n+1}} - a^{2^n} + 1$.

\item
Саша перемножил все делители натурального числа~$n$.
Федя увеличил каждый делитель на~1, а~потом перемножил результаты.
Федино произведение нацело делится на~Сашино.
Чему может быть равно $n$?

\item
Натуральные числа~$a$, $b$, $c$ таковы, что $a^2 + b^2 + c^2$ делится
на~$a + b + c$.
Докажите, что $a^5 + b^5 + c^5$ делится на~$a + b + c$.

\item
Существуют~ли 10 попарно взаимно простых натуральных чисел, сумма любых двух
из~которых делится на~квадрат некоторого простого числа?

\item
Натуральные числа $n$, $a$ и~$b$ таковы, что $(a - 1) (b - 1)$ не~делится
на~$n$, однако $a^{3} - 1$ и~$b^{5} - 1$ делятся на~$n$.
Докажите, что если для некоторого натурального~$k$ число $a^{k} b^{k} - 1$
делится на~$n$, то~$k$ делится на~$15$.

\item
Дано конечное множество простых чисел~$P$.
Докажите, что найдётся такое натуральное число~$x$ , что оно представляется
в~виде $x = a^{p} + b^{p}$ (с~натуральными $a, b$) при всех $p \in P$  
и~не~представляется в~таком виде для любого простого $p \notin P$.

\item
Докажите, что существует бесконечно много натуральных чисел, у~которых сумма
делителей~--- точный квадрат.

\end{problems}

