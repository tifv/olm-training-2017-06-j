% $date: 2017-06-10
% $timetable:
%   g8r1:
%     2017-06-11:
%       2:
%   g8r2:
%     2017-06-12:
%       1:
%   g8r3:
%     2017-06-10:
%       1:

\worksheet*{Китайская теорема об~остатках}

% $authors:
% - Лев Эдуардович Шабанов

\begin{problems}

\item
Найдите остаток от~деления натурального числа на~30, если при делении на~15 оно
дает остаток 7, а~при делении на~6 остаток 4.

\item
Пусть $a$ и~$b$~--- взаимно простые числа, а~$m$ и~$n$ дают одинаковые остатки
при делении на~$a$ и~при делении на~$b$.
Докажите, что $m - n$ делится на~$a b$.

\item
У~генерала не~более 5000 солдат.
Сначала он пытался построить их в~две равные шеренги, но~ему не~хватило одного
солдата.
Потом он пытался сделать тоже самое для трех, четырех, \ldots, десяти шеренг
и~все время у~него не~хватало одного солдата.
Докажите, что он сможет построить солдат в~11 равных шеренг.

\item
\subproblem
Лев загадал число от~0 до~60 и~сказал, что при делении на~5 оно даёт остаток 3,
при делении на~4 даёт 2 и~делится на~3.
Денис Владимирович уверен, что точно знает загаданное число.
Прав~ли он?
\\
\subproblem
А~если задумано число от~1 до~180?

\item\claim{Китайская теорема об~остатках (КТО)}
Пусть $m_{1}$, $m_{2}$, \ldots, $m_{n}$~--- попарно взаимно простые числа.
Тогда для любых целых $r_{1}, r_{2}, \ldots, r_{n}$ существует натуральное~$x$
такое, что
\[
    x \equiv_{m_{1}} r_{1}
, \quad
    x \equiv_{m_{2}} r_{2}
, \quad \ldots, \quad
    x \equiv_{m_{n}} r_{n}
. \]
%$x \equiv_{m_{1}} r_{1}$,\enspace
%$x \equiv_{m_{2}} r_{2}$,\enspace
%\ldots,\enspace
%$x \equiv_{m_{n}} r_{n}$.
Более того, среди чисел, не~превосходящих $m_{1} m_{2} \ldots m_{n}$, такое
число единственно.
\\
\subproblem
Докажите КТО для $n = 2$.
\\
\subproblem
Докажите КТО для произвольного $n$ методом математической индукции.
\\
\subproblem
Пусть $x$ и~$y$ дают одинаковые остатки при делении
на~$m_{1}$, \ldots, $m_{n}$.
Докажите, что $x - y \kratno m_{1} m_{2} \ldots m_{n}$ и~выведите отсюда КТО.
\\
\subproblem
С~помощью линейного представления НОД докажите, что такое $x$ существует при
$r_{1} = 1$, $r_{2} = 0$, $r_{3} = 0$, \ldots, $r_{n} = 0$.
Используя это соображение, докажите КТО.
\\
\subproblem
С~помощью функции Эйлера найдите какое-нибудь $x$, дающее остаток~1 при делении
на~$m_{1}$ и~остаток 0 при делении на~$m_{2}$, $m_{3}$, \ldots, $m_{n}$.
Докажите КТО.

\item
Попробуем избавится от~условия взаимной простоты в~условии КТО.
Пусть есть натуральные $m_{1}$, \ldots, $m_{n}$, и~для любых $i, j$ есть $x$
такое, что $x \equiv_{m_{i}} r_{i}$ и~$x \equiv_{m_{j}} r_{j}$.
Докажите, что тогда утверждение КТО выполнено.

\item
Пусть для натуральных $m_{1}$, \ldots, $m_{n}$, и~целых $r_{1}, \ldots, r_{n}$
нашлось натуральное~$x$ такое, что
\[
    x \equiv_{m_{1}} r_{1}
, \quad
    x \equiv_{m_{2}} r_{2}
, \quad \ldots, \quad
    x \equiv_{m_{n}} r_{n}
. \]
%$x \equiv_{m_{1}} r_{1}$,\enspace
%$x \equiv_{m_{2}} r_{2}$,\enspace
%\ldots,\enspace
%$x \equiv_{m_{n}} r_{n}$.
А~сколько существует таких $x$ среди чисел от~1 до~$m_{1} m_{2} \ldots m_{n}$?

\item
Пусть $m_{1}, m_{2}, \ldots, m_{n}$~--- попарно взаимно простые числа.
Тогда для любых целых $r_{1}, r_{2}, \ldots, r_{n}$ существует натуральное $x$
такое, что
\[
    x + 1 \equiv_{m_{1}} r_{1}
, \quad
    x + 2 \equiv_{m_{2}} r_{2}
, \quad \ldots, \quad
    x + n \equiv_{m_{n}} r_{n}
. \]
%$x + 1 \equiv_{m_{1}} r_{1}$,\enspace
%$x + 2 \equiv_{m_{2}} r_{2}$,\enspace
%\ldots,\enspace
%$x + n \equiv_{m_{n}} r_{n}$.

\item
Существует~ли в~сутках момент, когда расположенные на~общей оси часовая,
минутная и~секундная стрелки правильно идущих часов образуют попарно углы
в~$120^{\circ}$?

\end{problems}

