% $date: 2017-06-04
% $timetable:
%   g9r1:
%     2017-06-04:
%       1:

\worksheet*{Мультипликативные функции. Функция Эйлера}

% $authors:
% - Владимир Викторович Брагин

\begingroup
    \ifdefined\mathup
        \providecommand\eulerphi{\mathup{\mupphi}}\fi
    \ifdefined\upphi
        \providecommand\eulerphi{\upphi}\fi
    \providecommand\eulerphi{\phi}%

\subsubsection*{Теория}

\claim{Определение}
Функция~$f$, определенная на~натуральных числах, называется
\emph{мультипликативной,} если для любых натуральных взаимно простых $a$ и~$b$
выполнено $f(a b) = f(a) \cdot f(b)$.

\begin{problems}

\item
\subproblem
Докажите, что $\sigma(n)$, сумма делителей числа~$n$, является
мультипликативной функцией.
\\
\subproblem
Найдите явную формулу для $\sigma(n)$.

\item
\subproblem
Докажите, что $\eulerphi(n)$ является мультипликативной функцией.
\\
\subproblem
Найдите явную формулу для $\eulerphi(n)$.

\item
Отметим на~бумаге произвольным образом $\eulerphi(n)$ точек.
Каждой точке сопоставим какой-то обратимый остаток при делении на~$n$.
Проведем из~остатка~$k$ стрелочку в~остаток~$ka$.
\\
\subproblem
Убедитесь, что из~каждой точки выходит одна стрелочка, и~в~кажду точку входит
одна стрелочка.
\\
\subproblem
Поймите, что тогда все точки разбиваются на~циклические маршруты.
\\
\subproblem
Докажите, что у~всех циклических маршртутов одна и~та~же длина и~она делит
$\eulerphi(n)$.
\\
\subproblem
Выведите отсюда теорему Эйлера.

\item
Пусть
\( n = p_{1}^{\alpha_{1}} p_{2}^{\alpha_{2}} \ldots p_{k}^{\alpha_{k}} \).
Рассмотрим число
\[
    M
=
    \bigl[
        p_{1}^{\alpha_{1} - 1} (p_{1} - 1)
    , \,
        p_{2}^{\alpha_{2} - 1} (p_{2} - 1)
    , \, \ldots , \,
        p_{k}^{\alpha_{k} - 1} (p_{k} - 1)
    \bigr]
\, . \]
Докажите, что если $(a, n) = 1$, то~$a^{M} \equiv 1 \pmod{n}$.

\end{problems}


\subsubsection*{Задачи}

\begin{problems}

\item
Найдите все натуральные $n$, для которых
\\
\subproblem $\eulerphi(n) = 12$;
\quad
\subproblem $\eulerphi(n) = 14$;
\quad
\subproblem $\eulerphi(n) = \frac{n}{2}$;
\quad
\subproblem $\eulerphi(n) = \frac{n}{4}$.

\item
Докажите, что существует лишь конечное количество таких $n$, что
$\eulerphi(n) < 10^{9}$.

\item
Найдите остаток числа $4^{5^{6^7}}$ при делении на~$2017$.

\end{problems}


\subsubsection*{Сложные задачи}

\begin{problems}

\item
Назовем натуральное число \emph{совершенным,} если оно равно сумме всех своих
натуральных делителей, не~считая самого числа.
Докажите, что любое четное совершенное число имеет вид
$2^{p-1} \cdot (2^p - 1)$, где $p$ и~$2^p - 1$~--- простые числа.

\item
Докажите, что сумма квадратов делителей натурального числа меньше его квадрата.

\end{problems}

\endgroup % \def\eulerphi

