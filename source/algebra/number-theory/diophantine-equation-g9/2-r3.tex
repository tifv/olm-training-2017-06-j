% $date: 2017-06-11
% $timetable:
%   g9r3:
%     2017-06-11:
%       2:

\worksheet*{Уравнения в~целых числах~--- 2. Алгебраические преобразования}

% $authors:
% - Владимир Алексеевич Брагин

\begin{problems}

\item
Существуют~ли такие натуральные $a$ и~$b$, что
$a^2 b^2 + a^2 + b^2 + 1 = 2005$?

\item
Решите уравнение в~целых числах $n^2 + n = m^2$.

\item
Решите уравнение в~целых числах $n^2 + n - 8 = m^2$.

\item
Найдите все пары простых чисел $p$ и~$q$, для которых
выражение $p^2 + 5 p q + 4 q^2$ является квадратом натурального числа.

\item
Обозначим через $S(n)$ сумму цифр числа~$n$.
Существует~ли такое $n$, для которого $n^2 + S(n) + 1 = 2017^{2016}$?

\item
При каких натуральных~$n$ число $n^3 + 2 n^2 + 11$ является кубом натурального
числа?

\item
При каком основании системы счисления число 11010 является полным квадратом?

\item
Решите в~натуральных числах $1 + x + x^2 + x^3 = 2^y$.

%\item
%Найдите все упорядоченные тройки натуральных чисел $(a, b, c)$ такие, что все
%три числа $a^2 + 2 b + c$, $b^2 + 2 c + a$, $c^2 + 2 a + b$~--- точные
%квадраты.

\item
Найдите все такие натуральные~$n$, при которых число $5^{n} - 1$ можно
представить в~виде произведения четного числа последовательных натуральных
чисел.

\end{problems}

