% $date: 2017-06-09
% $timetable:
%   g9r3:
%     2017-06-09:
%       1:

\worksheet*{Теория чисел. Повторение + уравнения в~целых числах}

% $authors:
% - Владимир Алексеевич Брагин

\begin{problems}

\item
При каких $n$ число $7^{n} - 7$ делится на~$15$?

\item
Докажите, что если $(a, 2016) = 1$, то
\\
\subproblem $a^{48} - 1$;
\quad
\subproblem $a^{24} - 1$
\\
делится на~$2016$.

\item
Докажите, что $n^{561} - n$ делится на~$561$ при всех натуральных~$n$.

\item
Докажите, что если $n - 1$~--- это показатель числа $a$ по~модулю $n$,
то~$n$~--- простое число.

\item
Существуют~ли такие целые $x$, $y$, что $(x^2 + x + 1)^2 + (y^2 + y + 1)^2$
является квадратом натурального числа?

\item
Докажите, что следующие уравнения не~имеют решений в~натуральных числах:
\\
\subproblem $x^2 + y^2=2015$;
\\
\subproblem $x^2 + y^2 + z^2 = 7^{2017}$;
\\
\subproblem $x^2 + y^2 = 437 \cdot 2^{100}$;

\item
Найдите все такие натуральные $m$, $n$ и~$k$, что $m! + n! = 5^{k}$.

\item
Решите в~натуральных числах $3^{n} + 1 = 2^{m}$.

\item
Найдите все такие натуральные $m$ и~$k$, что $1^{m} + 2^{m} + 3^{m} = 6^{k}$.

\item
Решите в~натуральных числах $(3 x + 4 y) (25 x + 24 y) = 7^{2017}$.

\item
Существует~ли такое простое число $p$ и~натуральные $x$, $y$, $z$, что
$(12 x + 5) (12 y + 7) = p^{z}$?

\end{problems}

