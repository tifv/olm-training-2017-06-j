% $date: 2017-06-02
% $timetable:
%   g8r3:
%     2017-06-02:
%       1:

\worksheet*{НОД и~НОК}

% $authors:
% - Анна Николаевна Доледенок

\begingroup
    \def\gcd{\operatorname{\text{\rmfamily НОД}}}%
    \def\lcm{\operatorname{\text{\rmfamily НОК}}}%

Пусть $a$ и~$b$~--- два целых числа, не~равные одновременно нулю.
Рассмотрим все числа, на~которые делятся и~$a$ и~$b$ одновременно, т.\,е.
все общие делители $a$ и~$b$.
Выберем из~них наибольший и~назовем его \emph{наибольшим общим делителем}.
Будем обозначать наибольший общий делитель чисел $a$ и~$b$ через $\gcd(a, b)$
или, для краткости, $(a, b)$.
Наименьшим общим кратным чисел $a$ и~$b$
($\lcm(a, b)$ или, для краткости, $[a, b]$) называется наименьшее натуральное
число, которое делится на~$a$ и~$b$.

Аналогично определяется $\gcd$ и~$\lcm$ трех и~более чисел.
Например, пусть даны $n$~чисел $a_{1}, a_{2}, \ldots, a_{n}$.
Тогда $\gcd(a_{1}, a_{2}, \ldots, a_{n})$~--- это наибольшее число, на~которое
делится каждое из~данных $n$~чисел.

Если $\gcd(a, b) = 1$, то~числа $a$ и~$b$ называются \emph{взаимно простыми}.

\begin{problems}

\item
Докажите, что для любых натуральных чисел $a$ и~$b$ справедливо равенство
\(
    ab = \gcd(a, b) \cdot \lcm(a, b)
\).

\item
Верно~ли равенство
\(
    abc = \gcd(a, b, c) \cdot \lcm(a, b, c)
\)?

\item
Замените знаки ? в~выражении подходящими буквами (Д или К) и~докажите
получившееся равенство:
\[ \def\gcdlcm{\operatorname{\text{\rmfamily НО?}}}%
    \gcdlcm(a, b) \cdot \gcdlcm(b, c) \cdot \gcdlcm(c, a)
=
    \frac{a b c \cdot \gcdlcm(a, b, c)}{\gcdlcm(a, b, c)}
\, . \]

\item
\subproblem
Известно, что $\gcd(a, b, c) = 1$.
Верно~ли, что среди трех чисел $a, b, c$ найдутся два таких, что их НОД
равен 1?
\\
\subproblem
Верно~ли, что, если $\gcd(a, b) = \gcd(b, c) = d$, то~и~$\gcd(a, c) = d$?

\item
\subproblem
Может~ли так быть, что $\lcm(a, b) = a + b$?
\\
\subproblem
Может~ли так быть, что $\lcm(a, b, c) = a + b + c$?

\item
Какое наибольшее значение может принимать наибольший общий делитель чисел $a$
и~$b$, если известно, что $a b = 6000$?

\item
Денис посчитал НОК всех чисел от~1 до~1000, а~Лев~--- НОК всех чисел от~500
до~1000.
У~кого получилось большее число?

\item
$a$ и~$b$~--- натуральные числа.
Известно, что $a^2 + b^2$ делится на~$a b$.
Докажите, что $a = b$.

\item
$a, b, c$~--- натуральные числа, $\gcd(a, b, c) = 1$ и~$\frac{a b}{a - b} = c$.
Докажите, что $a - b$ является точным квадратом.

%\item
%Последовательность натуральных чисел $a_{i}$ такова, что
%$\gcd(a_{i}, a_{j}) = \gcd(i, j)$ для всех $i \neq j$.
%Докажите, что $a_{i} = i$ для всех натуральных $i$.

\item
Пусть $a$, $b$ и~$c$~--- попарно взаимно простые натуральные числа.
Найдите все возможные значения $\frac{(a + b) (b + c) (c + a)}{a b c}$, если
известно, что это число целое.

%\item
%Найдите все такие пары натуральных чисел $(x, y)$, что оба числа
%$x^3 + y$ и~$y^3 + x$ делятся на~$x^2 + y^2$.

\end{problems}

\endgroup % \def\gcd \def\lcm

