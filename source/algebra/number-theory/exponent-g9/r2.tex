% $date: 2017-06-08
% $timetable:
%   g9r2:
%     2017-06-08:
%       2:

\worksheet*{Вокруг теоремы Эйлера. Показатели}

% $authors:
% - Владимир Алексеевич Брагин

\begingroup
    \ifdefined\mathup
        \providecommand\eulerphi{\mathup{\mupphi}}\fi
    \ifdefined\upphi
        \providecommand\eulerphi{\upphi}\fi
    \providecommand\eulerphi{\phi}%

\begin{problems}

\item
При каких $n$ число $7^{n} - 7$ делится на~$15$?

\item
Докажите, что если $(a, 2016) = 1$, то
\\
\subproblem $a^{48} - 1$;
\quad
\subproblem $a^{24} - 1$
\\
делится на~$2016$.

\item
Докажите, что $n^{561} - n$ делится на~$561$ при всех натуральных $n$.

\item
Докажите, что $2^{32} + 1$ делится на~$641 = 2^4 + 5^4 = 5 \cdot 2^7 + 1$.

\end{problems}

Зафиксируем взаимно простые числа $a$ и~$n$.
Пусть $d$~--- наименьшее такое натуральное число, что $a^{d} - 1$ делится
на~$n$.
Число $d$ называется \emph{показателем~$a$ по~модулю~$n$}.

\begin{problems}

\item\claim{Важно запомнить}
Пусть $d$~--- показатель $a$ по~модулю~$n$.
\\
\subproblem
Докажите, что числа $a^0$, $a^1$, $a^2$, \ldots, $a^{d-1}$ дают разные остатки
при делении на~$n$.
\\
\subproblem
Пусть $a^l \equiv 1 \pmod{n}$.
Докажите, что $l \kratno d$.
\\
\subproblem
Докажите, что показатель любого числа по~модулю $n$ (взаимно простого с~$n$,
конечно) делит $\eulerphi(n)$.

\item
Какие значения могут принимать показатели по~модулю $13$?

\item
Докажите, что если $n - 1$~--- это показатель числа $a$ по~модулю $n$,
то~$n$~--- простое число.

\item
Докажите, что если $a > 1$, то~$n$ делит $\eulerphi(a^{n} - 1)$.

\item
\subproblem
Докажите, что любой нечетный простой делитель числа $a^{2^{k}} + 1$ имеет вид
$2^{k+1}x+1$.
\\
\subproblem
Докажите, что простых чисел вида $2^{2017} k + 1$ бесконечно много.

\item
Чему равен показатель $2017$ по~модулю $2^{2017}$?

\end{problems}

\endgroup % \def\eulerphi

