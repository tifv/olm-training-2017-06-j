% $date: 2017-06-06
% $timetable:
%   g9r3:
%     2017-06-06:
%       1:

\worksheet*{Стрелочки ведут по~кругу. Показатели}

% $authors:
% - Владимир Алексеевич Брагин

\begingroup
    \ifdefined\mathup
        \providecommand\eulerphi{\mathup{\mupphi}}\fi
    \ifdefined\upphi
        \providecommand\eulerphi{\upphi}\fi
    \providecommand\eulerphi{\phi}%

Обозначим за~$\eulerphi(n)$ количество чисел от~$1$ до~$n$, взаимно простых
с~$n$.

\claim{Теорема Эйлера}
Если $a$ и~$n$~--- взаимно простые числа,
то~$a^{\eulerphi(n)} \equiv 1 \pmod n$.

Зафиксируем взаимно простые числа $a$ и~$n$.
Пусть $d$~--- наименьшее такое натуральное число, что $a^{d} - 1$ делится
на~$n$.
Число $d$ называется \emph{показателем~$a$ по~модулю~$n$}.

\begin{problems}

\itemy{0}\emph{(скоро разберем)}\enspace
Аналогично последней задаче из~прошлого листа докажите теорему Эйлера.

\item\claim{Важно запомнить}
Пусть $d$~--- показатель $a$ по~модулю~$n$.
\\
\subproblem
Докажите, что числа $a^0$, $a^1$, $a^2$, \ldots, $a^{d-1}$ дают разные остатки
при делении на~$n$.
\\
\subproblem
Пусть $a^l \equiv 1 \pmod{n}$.
Докажите, что $l \kratno d$.

\item
Какие значения могут принимать показатели по~модулю $13$?

\item
Докажите, что показатель любого числа по~модулю $n$ (взаимно простого с~$n$,
конечно) делит $\eulerphi(n)$.

\item
Докажите, что если $a > 1$, то~$n$ делит $\eulerphi(a^{n} - 1)$.

\item
\subproblem
Пусть $p$~--- нечетное простое число.
Докажите, что любой простой делитель числа $a^{p} - 1$ или делит $a - 1$, или
имеет вид $2 p x + 1$.
\\
\subproblem
Выведите отсюда, что простых чисел вида $2 p k + 1$ бесконечно много.

\item
\subproblem
Докажите, что любой нечетный простой делитель числа $a^{2^{k}} + 1$ имеет вид
$2^{k+1} x + 1$.
\\
\subproblem
Докажите, что простых чисел вида $2^{k+1} x + 1$ бесконечно много.

\item
Докажите, что если $n > 1$, то~$2^{n} - 1$ не~делится на~$n$.

\end{problems}

\endgroup % \def\eulerphi

