% $date: 2017-06-08
% $timetable:
%   g9r1:
%     2017-06-08:
%       1:

\worksheet*{Показатели}

% $authors:
% - Владимир Алексеевич Брагин

\begingroup
    \ifdefined\mathup
        \providecommand\eulerphi{\mathup{\mupphi}}\fi
    \ifdefined\upphi
        \providecommand\eulerphi{\upphi}\fi
    \providecommand\eulerphi{\phi}%

\begin{problems}

\item\claim{Важно запомнить}
Пусть $d$~--- показатель $a$ по~модулю~$n$.
\\
\subproblem
Докажите, что числа $a^0$, $a^1$, $a^2$, \ldots, $a^{d-1}$ дают разные остатки
при делении на~$n$.
\\
\subproblem
Пусть $a^l \equiv 1 \pmod{n}$.
Докажите, что $l \kratno d$.

\end{problems}

\subsubsection*{Кто-то видел, кто-то нет, но~полезно знать}

\begin{problems}

\item
Докажите, что если показатель числа $a$ по~модулю $n$ равен $n - 1$, то~$n$~---
простое число.

\item
Докажите, что если $a > 1$, то~$n$ делит $\eulerphi(a^{n} - 1)$.

\item
\subproblem
Пусть $p$~--- нечетное простое число.
Докажите, что любой простой делитель числа $a^{p} - 1$ или делит $a - 1$, или
имеет вид $2 p x + 1$.
\\
\subproblem
Выведите отсюда, что для любого простого $p$ среди чисел вида $2 p k + 1$
бесконечно много простых.

\item
\subproblem
Докажите, что любой нечетный простой делитель числа $a^{2^{k}} + 1$ имеет вид
$2^{k+1} x + 1$.
\\
\subproblem
Докажите, что для любого $k$ простых чисел вида $2^{k+1} x + 1$ бесконечно
много.

\end{problems}

\subsubsection*{Разное}

\begin{problems}

\item
Чему равен показатель $2017$ по~модулю $2^{2017}$?

\item
Докажите, что если $n > 1$, то~$2^{n} - 1$ не~делится на~$n$.

\item
Найдите все такие пары простых чисел $p$ и~$q$, что $p^{p} + q^{q} + 1$ делится
на~$p q$.

\item
Найдите все тройки простых чисел $p$, $q$ и~$r$ такие, что
$q^r + 1 \kratno p$, $r^p + 1 \kratno q$, $p^q + 1 \kratno r$.

%\item
%Найдите все такие пары простых чисел $p$ и~$q$, что число $5^{p} + 5^{q}$
%делится на~$p q$.

\end{problems}

\endgroup % \def\eulerphi

