% $date: 2017-06-11
% $timetable:
%   g8r3:
%     2017-06-11:
%       1:

\worksheet*{Натуральные уравнения}

% $authors:
% - Лев Эдуардович Шабанов

\begin{problems}

\item\emph{Задача о~размене монет.}
Пусть в~некоторой стране есть монеты достоинством $a$ и~$b$ тугриков, где $a$
и~$b$ взаимно просты.
Докажите, что:
\\
\subproblem
любую сумму, не~меньшую, чем $(a - 1) (b - 1)$ тугриков, можно заплатить без
сдачи;
\\
\subproblem
если $x$~тугриков можно заплатить без сдачи, то~$a b - a - b - x$ тугриков
заплатить без сдачи нельзя;
\\
\subproblem
если $x < (a - 1) (b - 1)$ тугриков нельзя заплатить без сдачи,
то~$ab - a - b - x$~--- можно.
\\
\subproblem
Не~пользуясь соображениями двух предыдущих пунктов докажите, что сумм, которые
меньше, чем $(a - 1) (b - 1)$ и~которые можно заплатить без сдачи,
ровно $\frac{1}{2} (a - 1) (b - 1)$. 

\item
При каких $N$ уравнение $5 x + 8 y = N$ имеет $6$~решений в~натуральных числах? 

\item
При каком наибольшем $N$ уравнение $99 x + 100 y + 101 z = N$ имеет
единственное решение в~натуральных числах $x, y, z$?

%\item
%При каком наименьшем $n$ квадрат $n \times n$ можно разрезать на~квадраты
%$40 \times 40$ и~$49 \times 49$ так, чтобы квадраты обоих видов присутствовали?

\end{problems}

