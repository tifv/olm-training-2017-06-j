% $date: 2017-06-03
% $timetable:
%   g8r1:
%     2017-06-03:
%       1:

\worksheet*{Алгоритм Евклида}

% $authors:
% - Анна Николаевна Доледенок

\begingroup
    \def\gcd{\operatorname{\text{\rmfamily НОД}}}%
    \def\lcm{\operatorname{\text{\rmfamily НОК}}}%
    \def\abs#1{\lvert #1 \rvert}%

$\gcd(a, b) = \gcd(a, b - a)$.
\par
Алгоритм Евклида~--- правило, которое позволяет по~двум натуральным
числам $a$ и~$b$ найти $\gcd(a, b)$.
Если имеются два натуральных числа $a > b > 0$, то~сначала делим $a$ на~$b$,
и~получаем остаток~$r_{1}$, который меньше, чем $b$.
Затем делим число~$b$ на~$r_{1}$, находим остаток~$r_{2}$, который меньше, чем
$r_{1}$.
Далее делим число~$r_{1}$ на~$r_{2}$, находим остаток~$r_{3}$, меньший, чем
$r_{2}$, и~так далее.
Указанный процесс обязательно закончится, поскольку каждый следующий остаток
меньше предыдущего, а~все остатки~--- неотрицательные числа.
В~конце концов какой-то остаток~$r_{n-1}$ разделится на~остаток~$r_{n}$ нацело,
без остатка.
Последний остаток $r_{n}$ и~есть $\gcd(a, b)$, так как
\[
    r_{n} = \gcd(r_{n}, r_{n-1}) = \gcd(r_{n-1}, r_{n-2})
= \ldots =
    \gcd(r_{2}, r_{1}) = \gcd(r_1, b) = \gcd(b, a)
\, . \]

\begin{problems}

\item
На~доске написаны числа $24$ и~$36$.
За~ход разрешается дописать еще одно натуральное число~--- разность любых двух
имеющихся на~доске чисел, если она еще не~встречалась.
Проигрывает тот, кто не~может сделать ход.
Кто выиграет при правильной игре?

\item
Найдите НОД всех чисел, в~записи каждого из~которых все цифры
$1, 2, \ldots, 9$ использованы по~одному разу.

\item
Найдите $\gcd(11{\ldots}1, 11{\ldots}1)$, где в~первом числе $n$ единиц,
а~во~втором $m$.

\item
Докажите, что $\gcd(a^{m} - 1, a^{n} - 1) = a^d - 1$, где $d = \gcd(m, n)$.

\item
\subproblem
Найдите НОД двух чисел Ферма $2^{2^{n}} + 1$ и~$2^{2^{m}} + 1$.
(Докажите, что число $2^{2^n}-1$ имеет хотя~бы $n$ различных простых
делителей.)
\\
\subproblem
Найдите НОД двух чисел Фибоначчи $F_{n}$ и~$F_{m}$.

\item
Функция $f(a, b)$, определенная для всех натуральных $a$ и~$b$, обладает
следующими тремя свойствами:
\\
\textit{(1)} $f(a, a) = a$;
\\
\textit{(2)} $f(a, b) = f(b, a)$;
\\
\textit{(3)} $(a + b) \cdot f(a, b) = b \cdot f(a, a + b)$.
\\
Докажите, что $f(a, b) = \lcm(a, b)$.

\item
Пусть $a$, $b$, $c$~--- натуральные числа.
Могут~ли $\lcm(a, b)$ и~$\lcm(a + c, b + c)$ быть равны?

%\item
%Даны шесть натуральных чисел.
%Для каждой пары чисел посчитали их НОД.
%Могут~ли среди этих НОДов встречаться все натуральные числа от~1 до~15?

%\item
%Можно~ли вместо звездочек вставить в~выражение
%$\lcm(*, *, *) - \lcm(*, *, *) = 2009$
%в~некотором порядке шесть последовательных натуральных чисел так, чтобы
%равенство стало верным?

\item
На~доске написаны натуральные числа $a_{1}, \ldots, a_{k}$ такие, что
$\gcd(a_{1}, \ldots, a_{k}) = 1$.
Разрешается взять два числа и~из~большего вычесть меньшее.
Докажите, что такими операциями можно получить число~$1$.

\item
На~доске написаны два различных натуральных числа $a$ и~$b$.
Меньшее из~них стирают, и~вместо него пишут число $\frac{ab}{\abs{a - b}}$
(которое может уже оказаться нецелым).
С~полученной парой чисел делают ту~же операцию и~т.\,д.
Докажите, что в~некоторый момент на~доске окажутся два равных натуральных
числа.

\item
Аня нашла себе интересное занятие.
Она написала на~доске две единички, потом между ними написала их сумму.
Ее это так захватило, что она продолжила: брала ряд чисел, который у~нее
получился на~предыдущем шаге, и~между двумя соседними числами писала их сумму
(старые числа при этом не~стирала).
Сколько раз она выписала произвольное число~$n$?

\end{problems}

\endgroup % \def\gcd \def\lcm

