% $date: 2017-06-04
% $timetable:
%   g8r3:
%     2017-06-04:
%       1:

\worksheet*{Линейное представление НОД}

% $authors:
% - Анна Николаевна Доледенок

\begingroup
    \def\gcd{\operatorname{\text{\rmfamily НОД}}}%
    \def\lcm{\operatorname{\text{\rmfamily НОК}}}%

\begin{problems}

\item
Сколько существует пар целых чисел $(x, y)$ таких, что $a x + b y = (a, b)$?

\item
На~прямой сидит блоха, которая может прыгать либо на~15 сантиметров влево, либо
на~21 сантиметр вправо.
В~каких точках прямой она может побывать?

\item
\subproblem
Пусть $a u + b v = 1$, где $a$, $b$, $u$ и~$v$~--- некоторые целые числа.
Докажите, что $\gcd(a, b) = 1$.
\\
\subproblem
Пусть $a u + b v = 2$.
Верно~ли, что $\gcd(a, b) = 2$?

%\item
%Докажите, что для любого набора натуральных чисел
%$a_{1}$, $a_{2}$, \ldots, $a_{n}$
%найдутся целые числа $x_{1}$, $x_{2}$, \ldots, $x_{n}$ такие, что
%\(
%    a_{1} x_{1} + a_{2} x_{2} + \ldots + a_{n} x_{n}
%=
%    (a_{1}, a_{2}, \ldots, a_{n})
%\).

\item
У~Пети есть уголок с~углом $82^{\circ}$.
Сколько раз ему нужно воспользоваться этим уголком, чтобы нарисовать угол
в~$2^{\circ}$?

\item
В~банке 500 долларов.
Разрешаются две операции: взять из~банка 300 долларов или положить в~него
198 долларов.
Эти операции можно проводить много раз, при этом, однако, никаких денег, кроме
тех, что первоначально лежат в~банке, нет.
Какую максимальную сумму можно извлечь из~банка и~как это сделать?

%\item
%На~острове Мумбо-Юмбо есть монеты достоинством в~252, 315 и~350 тугриков.
%Какую наименьшую сумму можно оплатить на~этом острове, если в~любом магазине
%на~сдачу есть неограниченное число монет.
%Какое наименьшее число монет может участвовать в~оплате этой суммы?

\item
В~классе химии имеются 25~пробирок объема 1, 2, \ldots, 25~мл.
Когда химики стали собираться в~ЛМШ, оказалось, что у~них осталось мало места,
и~они могут взять только набор из~10 пробирок.
Химики хотят, чтобы с~помощью любых двух пробирок из~набора можно было отмерить
1~мл.
Сколькими способами можно составить такой набор?

\item
У~ресторана, где проводится торжественный вечер, есть столы на~12
и~на~7 человек.
На~торжественный вечер организаторы хотят пригласить лучших работников, однако
помимо работников на~этот вечер хочет прийти неизвестное число гостей
из~городской администрации.
По~соображениям безопасности, все столы, использованные для вечера должны быть
заняты полностью.
Какое наименьшее число работников должны пригласить организаторы, чтобы
независимо от~количества людей из~администрации вечер не~сорвался?

\item
Остап Бендер организовал в~городе Фуксе раздачу слонов населению.
На~раздачу явились 28 членов профсоюза и~37 не~членов, причём Остап раздавал
слонов поровну всем членам профсоюза и~поровну~--- не~членам.
Оказалось, что существует лишь один способ такой раздачи (так, чтобы раздать
всех слонов).
Какое наибольшее число слонов могло быть у~О.\,Бендера?

\item
Бильярдный стол имеет форму прямоугольника $m \times n$, в~углах которого
расположены лунки.
Из~левого нижнего угла выпускают шар под углом $45^\circ$ к~сторонам стола.
%\\
%\subproblem
Сколько раз шарик ударится о~края стола прежде чем упасть в~лунку?
%\\
%\subproblem
%Через какие точки стола пролетит шар прежде чем упасть в~лунку?

\end{problems}

\endgroup % \def\gcd \def\lcm

