% $date: 2017-06-03
% $timetable:
%   g9r3:
%     2017-06-03:
%       2:

\worksheet*{Деление сравнений. Малая теорема Ферма}

% $authors:
% - Владимир Алексеевич Брагин

\begin{problems}

\item\claim{О~сокращениях в~сравнениях по~модулю}
Докажите, что если $a$ и~$n$ взаимно просты,
и~$ax \equiv ay \pmod{n}$, то~$x \equiv y \pmod{n}$.

\item
Целые числа $a$ и~$b$ таковы, что $6 a + 34 b \kratno 31$.
Докажите, что $3 b - 25 a \kratno 31$ и~$23 a - 4 b \kratno 31$.

\item
Даны взаимно простые натуральные числа $n$ и~$a$.
Докажите, что в~арифметической прогрессии длины~$n$ с~разностью~$a$
найдется ровно одно число, делящееся на~$n$.

\item
Решите сравнения:
\\
\subproblem
$5 x \equiv 6 \pmod {11}$.
\\
\subproblem
$x^2 - 5 x + 6 \equiv 0 \pmod {13}$.
\\
\subproblem
$x^2 + 5 x - 6 \equiv 0 \pmod {21}$.
\\
\subproblem
$x^2 - 7 x \equiv 67 \pmod {77}$.

\end{problems}

\claim{Малая теорема Ферма}
Если $p$~--- простое число, а~$a$ не~делится на~$p$, то~$a^{p-1} - 1$ делится на~$p$.

\emph{Альтернативная формулировка.}
Если $p$~--- простое число, то~$a^{p} - a$ делится на~$p$.

\begin{problems}

\item
\subproblem
Пусть $a$ не~делится на~$p$.
Докажите, что среди чисел $2 \cdot a, 2 \cdot a, \ldots, (p - 1) \cdot a$ все
ненулевые остатки при делении на~$p$ содержатся по~одному разу.
\\
\subproblem
Выведите отсюда, что $(p - 1)! \equiv (p-1)! \cdot a^{p-1} \pmod{p}$ и~докажите
малую теорему Ферма.

\item
\subproblem
Какой остаток дает число $3^{2001}$ при делении на~$101$?
\\
\subproblem
Какой остаток может давать число $a^{50}$ при делении на~$101$?

\item
Докажите, что число $2^{35} + 3^{35} + 6^{35} - 1$ делится на~$37$.

\item
\emph{Другое очень поучительное доказательство.}
Зафиксируем ненулевой остаток $a$ при делении на~$p$.
Отметим на~бумаге произвольным образом $p - 1$ точек.
Каждой точке сопоставим какой-то ненулевой остаток при делении на~$p$.
Проведем из~остатка~$k$ стрелочку в~остаток~$ka$.
\\
\subproblem
Убедитесь, что из~каждой точки выходит одна стрелочка, и~в~каждую точку входит
одна стрелочка.
\\
\subproblem
Поймите, что тогда все точки разбиваются на~циклические маршруты.
\\
\subproblem
Докажите, что у~всех циклических маршртутов одна и~та~же длина и~она делит
$p - 1$.
\\
\subproblem
Выведите отсюда малую теорему Ферма.

\end{problems}

