% $date: 2017-06-04
% $timetable:
%   g9r2:
%     2017-06-04:
%       2:

\worksheet*{Теорема Эйлера. Функция Эйлера}

% $authors:
% - Владимир Алексеевич Брагин

\begingroup
    \ifdefined\mathup
        \providecommand\eulerphi{\mathup{\mupphi}}\fi
    \ifdefined\upphi
        \providecommand\eulerphi{\upphi}\fi
    \providecommand\eulerphi{\phi}%

\claim{Малая теорема Ферма}
Если $p$~--- простое число, а~$a$ не~делится на~$p$,
то~$a^{p-1} \equiv 1 \pmod{p}$.

\claim{Теорема Эйлера}
Если $a$ и~$n$~--- взаимно простые числа,
то~$a^{\eulerphi(n)} \equiv 1 \pmod n$.

\begin{problems}

\item
Зафиксируем ненулевой остаток $a$ при делении на~$p$.
Отметим на~бумаге произвольным образом $p - 1$ точек.
Каждой точке сопоставим какой-то ненулевой остаток при делении на~$p$.
Проведем из~остатка~$k$ стрелочку в~остаток~$ka$.
\\
\subproblem
Убедитесь, что из~каждой точки выходит одна стрелочка, и~в~каждую точку входит
одна стрелочка.
\\
\subproblem
Поймите, что тогда все точки разбиваются на~циклические маршруты.
\\
\subproblem
Докажите, что у~всех циклических маршртутов одна и~та~же длина и~она делит
$p - 1$.
\\
\subproblem
Выведите отсюда малую теорему Ферма.

\item
Аналогично докажите теорему Эйлера.

\item
Пусть
\( n = p_{1}^{\alpha_{1}} p_{2}^{\alpha_{2}} \ldots p_{k}^{\alpha_{k}} \).
Рассмотрим число
\[
    M
=
    \bigl[
        p_{1}^{\alpha_{1} - 1} (p_{1} - 1)
    , \,
        p_{2}^{\alpha_{2} - 1} (p_{2} - 1)
    , \, \ldots , \,
        p_{k}^{\alpha_{k} - 1} (p_{k} - 1)
    \bigr]
\, . \]
Докажите, что если $(a, n) = 1$, то~$a^{M} \equiv 1 \pmod{n}$.

\item
\subproblem
Докажите, что $n^{84} - n^{4}$ делится на~$20400$ для любого натурального $n$.
\\
\subproblem
Можно~ли вместо $20400$ доказать для какого-то большего числа?

\end{problems}

\claim{Определение}
Функция~$f$, определенная на~натуральных числах, называется
\emph{мультипликативной,} если для любых натуральных взаимно простых $a$ и~$b$
выполнено $f(a b) = f(a) \cdot f(b)$.

\begin{problems}

\item
\subproblem
Докажите, что $\eulerphi(n)$ является мультипликативной функцией.
\\
\subproblem
Найдите явную формулу для $\eulerphi(n)$.

\item
Найдите все натуральные $n$, для которых
\\
\subproblem $\eulerphi(n) = 12$;
\quad
\subproblem $\eulerphi(n) = 14$;
\quad
\subproblem $\eulerphi(n) = \frac{n}{2}$;
\quad
\subproblem $\eulerphi(n) = \frac{n}{4}$.

\item
Докажите, что существует лишь конечное количество таких $n$, что
$\eulerphi(n) < 10^{9}$.

\end{problems}

\endgroup % \def\eulerphi

