% $date: 2017-06-11
% $timetable:
%   g8r3:
%     2017-06-11:
%       2:

\worksheet*{Индукция в неравенствах}

% $authors:
% - Алексей Сергеевич Семченков

% $build$matter[print]: [[.], [.]]
% $build$matter[print,author]: [[.]]
% $build$style[print,author]:
% - .[resize-to]

\begin{problems}

\item
Докажите, что есть такое $n$, что $1{,}001^{n} \geq 2017$.

\item
Определим $L_{n}$ как 
\(
    \frac{1}{\sqrt{1}} + \frac{1}{\sqrt{2}} + \ldots + \frac{1}{\sqrt{n}}
\).
Докажите, что $L_{n} \geq 2 \sqrt{n + 3} - 3$. 

\item
Определим гармонический ряд $H_{n}$ как 
\(
    1 + \frac{1}{2} + \frac{1}{3} + \ldots + \frac{1}{n}
\).
\\
\subproblem
Докажите, что $H_{7} \geq 1{,}5$.
\\
\subproblem
Докажите, что $H_{15} \geq 2$.
\\
\subproblem
Докажите, что найдется такой $n$, что $H_{n} \geq 2017$.

\item
Положим величину $T_{n}$ равной
\(
    \left( 1 + \frac{1}{1} \right)
    \left( 1 + \frac{1}{3} \right)
    \ldots
    \left( 1 + \frac{1}{2n - 1} \right)
\).
Докажите, что $T_{n} \geq \sqrt{2 n + 1}$.

\item
Сформулируем неравенство средних для $n$~переменных:\\
для любых положительных $x_1, x_2, \ldots, x_n$ верно:
\[
    \frac{x_{1} + x_{2} + \ldots + x_{n}}{n}
\geq
    \sqrt[n]{x_{1} x_{2} \ldots x_{n}}
\]
\subproblem
Докажите неравенство, когда $n$ --- степень двойки.
\\
\subproblem
Докажите, что если неравенство верно для любых $n$~переменных, то оно верно
и~для любых $n - 1$ переменных.
\\
\subproblem
Докажите, что неравенство выполнено для любых $n$~переменных.

\item
Докажите, что для любого $n$ выполнено неравенство
\\[0.5ex]
\subproblem
\( \displaystyle
    \frac{1}{1 \cdot 2} + \frac{1}{2 \cdot 3} + \ldots
    + \frac{1}{n \cdot (n + 1)}
<
    1
\);
\\[1.0ex]
\subproblem
\( \displaystyle
    1 + \frac{1}{4} + \frac{1}{9} + \cdots + \frac{1}{n^2}
<
    2
\).

\end{problems}

