% $date: 2017-06-01
% $timetable:
%   g9r1:
%     2017-06-01:
%       3:
%   g9r2:
%     2017-06-01:
%       3:
%   g9r3:
%     2017-06-01:
%       3:

\worksheet*{Письменная вступительная олимпиада}

% $authors:
% - Владимир Алексеевич Брагин
% - Владислав Викторович Новиков
% - Аскар Флоридович Назмутдинов
% - Леонид Андреевич Попов

\begin{problems}

\item
В~остроугольном треугольнике $ABC$ высоты $A A_1$ и~$C C_1$ пересекаются
в~точке~$H$, а~биссектрисы $A A_2$ и~$C C_2$ в~точке~$I$.
Оказалось, что $\angle AHC = \angle AIC$.
Найдите угол $ABC$.

\item
6~команд играют между собой чемпионат.
В~каждом туре встречаются какие-то 3~пары команд, не~игравшие между собой
ранее.
В~какой-то момент выяснилось, что очередной тур провести невозможно.
Определите наименьший номер тура, в~котором это могло произойти.

\item
Найдите все натуральные $x$, $y$, $z$ такие, что $4^{x} + 3^{y} = z^{2}$.

\item
На~боковых сторонах $AB$ и~$CD$ трапеции $ABCD$ выбраны точки $X$ и~$Z$
соответственно.
Отрезки $CX$ и~$BZ$ пересекаются в~точке~$Y$.
Оказалось, что пятиугольник $AXYZD$~--- вписанный.
Докажите, что $AY = DY$.

\item
Вася написал на~доске 99 вещественных чисел.
Пусть $A$~--- квадрат суммы Васиных чисел, а~$B$~--- сумма их квадратов,
умноженная на~100.
Докажите, что Петя может поменять знаки у~некоторых Васиных чисел так, чтобы
квадрат суммы всех чисел стал не~больше, чем $B - A$.

\item
Дан клетчатый квадрат~$n \times n$, в~котором стерли все клетки выше главной
диагонали (идущей из~левого верхнего угла в~правый нижний).
В~каждой клетке оставшейся фигуры записывают~$0$ или~$1$.
При этом, если в~какой-то клетке написана единица, то~и~в~соседних с~ней
по~стороне слева и~сверху также должна стоять единица.
Сколькими способами это можно сделать?

\end{problems}

