% $date: 2017-06-01
% $timetable:
%   g8r1:
%     2017-06-01:
%       3:
%   g8r2:
%     2017-06-01:
%       3:
%   g8r3:
%     2017-06-01:
%       3:

\worksheet*{Устная вступительная олимпиада}

% $authors:
% - Андрей Юрьевич Кушнир
% - Вадим Дмитриевич Лучкин

\subsection*{Довывод}

\begin{problems}

\item
Можно~ли шахматную доску $8 \times 8$ разрезать на~$15$ горизонтальных
и~$17$ вертикальных доминошек?

\item
Петя перемножил все нечётные натуральные числа, меньшие $1000$.
Какая цифра стоит в~произведении в~разряде десятков?

\item
Две дворовые команды играют в~футбол до~$10$~голов
(игра прекращается, когда одна из~команд забьёт $10$ голов).
В~процессе игры ведётся протокол, в~который вносится счёт после каждого гола.
Сколько различных протоколов может получиться?

\item
В~треугольнике $ABC$ проведены биссектрисы углов $B$ и~$C$.
Основания перпенидикуляров, опущенных из~вершины $A$ на~эти биссектрисы,
обозначены $P$ и~$Q$.
Докажите, что $PQ$ и~$BC$ параллельны.

\item
Андрей и~Борис играют в~следующую игру: они по~очереди проводят прямые
плоскости, проходящие через фиксированную точку~$S$.
Запрещается проводить одну и~ту~же прямую два раза.
Проигрывает тот, после чьего хода угол между какими-то из~проведённых прямых
окажется меньше $1^{\circ}$.
Первым ходит Андрей.
Может~ли кто-либо из~игроков обеспечить себе победу вне зависимости от~игры
соперника?
Если да, то~кто?

\end{problems}

