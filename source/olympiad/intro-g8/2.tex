% $date: 2017-06-01
% $timetable:
%   g8r1:
%     2017-06-01:
%       3:
%   g8r2:
%     2017-06-01:
%       3:
%   g8r3:
%     2017-06-01:
%       3:

% $authors:
% - Андрей Юрьевич Кушнир
% - Вадим Дмитриевич Лучкин

% $matter[-contained,no-header]:
% - verbatim: \worksheet*{Устная вступительная олимпиада}
% - verbatim: \setproblem{5}
% - .[contained]

\subsection*{Вывод}

\begin{problems}

\item
Натуральные числа от~$1$ до~$100$ раскрасили в~три цвета.
Докажите, что найдутся два одноцветных числа, разность между которыми является
квадратом натурального числа.

\item
В~равнобедренном прямоугольном треугольнике $ABC$ на~гипотенузе~$AB$ взяты
точки $M$ и~$N$ ($N$ между $M$ и~$B$) такие, что $\angle MCN = 45^{\circ}$.
Докажите, что $MN^2 = AM^2 + NB^2$.

\item
На~острове Кокос проживает $2017$ аборигенов, каждый из~которых либо всегда
говорит правду (рыцарь), либо всегда обманывает (лжец), причем не~все из~них
лжецы.
Путешественник хочет узнать количество рыцарей на~этом острове.
Ему разрешено один раз в~день собирать на~берегу любую группу островитян,
каждый из~которых назовет количество рыцарей среди собравшихся.
За~какое наименьшее число дней путешественник сможет выяснить точное число
рыцарей?

\end{problems}

