% $date: 2017-06-02
% $timetable:
%   g9r1:
%     2017-06-02:
%       1:

\worksheet*{Перечислительная комбинаторика}

% $authors:
% - Аскар Флоридович Назмутдинов

% $build$matter[print]: [[.], [.]]
% $build$matter[print,author]: [[.]]
% $build$style[print,author]:
% - .[resize-to]

\begin{problems}

\item
В~городе $n$~городов, соединённых попарно дорогами.
Скольким количеством способов на~каждой из~дорог можно выбрать направление
движения так, чтобы существовал хотя~бы один циклический маршрут
(возможно, проходящий не~по~всем городам)?

\item
Сколькими способами в~классе из~$20$~человек выбрать две непересекающиеся
группы?

\item
\subproblem
Сколькими способами можно выстроить класс из~$20$~человек в~хоровод, чтобы Петя
был на~одинаковом расстоянии от~Кати и~Маши?
\\
\subproblem
Сколькими способами можно выстроить класс из~$20$~человек в~очередь, чтобы
Петя находился на~одного человека ближе к~Кате, чем к~Маше?

\item
Назовем девятизначное число \emph{хорошим,} если в~нем можно переставить одну
цифру на~другое место и~получить девятизначное число, в~котором цифры идут
строго по~убыванию.
Сколько всего хороших чисел?

\item
Имеется 20~человек~--- 10~юношей и~10~девушек.
Сколько существует способов составить компанию, в~которой было~бы одинаковое
число юношей и~девушек?

\item
Сколько существует десятизначных чисел, в~которых цифры не~возрастают?

\item
Сколькими способами можно выставить от~$0$ до~$100$ баллов за~ЕГЭ
отличнице Кате и~двоечнику Пете, чтобы у~Кати было баллов больше, чем у~Пети,
и~не~меньше $50$?

\item
Сколькими способами можно разрезать палку длины $2017$ на~$3$~части целой длины
так, чтобы потом из~этих частей можно было~бы сложить треугольник?

\item
В~обществе из~$n$~членов каждое непустое подмножество считается комиссией.
В~каждой комиссии нужно выбрать одного из~членов председателем, соблюдая
правило: если комиссия~$C$ является объединением нескольких меньших комиссий
(возможно, пересекающихся), то~председателем $C$ должен быть один
из~председателей этих меньших комиссий.
Cколькими способами можно выбрать председателей?

\item
Назовем натуральное число \emph{интересным,} если оно~--- степень тройки или
представимо в~виде суммы различных степеней тройки.
Интересные числа занумеровали по~возрастанию.
Найдите сотое интересное число.

\item
Петя хочет выписать все возможные последовательности из~100 натуральных
чисел, в~каждой из~которых хотя~бы раз встречается тройка, а~любые два соседних
члена различаются не~больше, чем на~1.
Сколько последовательностей ему придётся выписать?

\end{problems}

