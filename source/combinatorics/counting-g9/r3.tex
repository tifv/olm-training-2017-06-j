% $date: 2017-06-03
% $timetable:
%   g9r3:
%     2017-06-03:
%       1:

\worksheet*{Перечислительная комбинаторика}

% $authors:
% - Владислав Викторович Новиков

\begin{problems}

\item
Сколькими способами можно расставить $20$~детям в~классе баллы за~ЕГЭ
(от~$0$ до~$100$), чтобы нашелся ученик, набравший больше $90$~баллов?

\item
Сколькими способами можно расставить в~очередь $20$~детей, чтобы между Петей
и~Катей стояло ровно $4$ человека?

\item
Сколькими способами можно выставить от~$0$ до~$100$ баллов за~ЕГЭ
отличнице Кате и~двоечнику Пете, чтобы у~Кати было баллов больше, чем у~Пети и
\\
\subproblem
у~Кати было не~меньше $50$?
\\
\subproblem
у~Кати и~Пети в~сумме было не~меньше $50$?

\item
Сколькими способами в~классе из~20 человек можно выбрать две непересекающиеся
группы (группы неразличимы и~могут быть пустыми)?

\item
На~полке стоят 12~книг.
Сколькими способами можно выбрать из~них 5~книг, никакие две из~которых
не~стоят рядом?

\item
Сколькими способами можно выстроить класс из~$20$ человек в~хоровод, чтобы Петя
был на~одинаковом расстоянии от~Кати и~Маши?

\item
У~Деда Мороза 100 конфет.
Он хочет раздать их 20 детям так, чтобы каждый получил хотя~бы по~одной
конфете.
Сколькими способами он может это сделать?

\item
Имеется 20~человек~--- 10~юношей и~10 девушек.
Сколько существует способов составить компанию, в~которой было~бы одинаковое
число юношей и~девушек?

\item
Назовем девятизначное число \emph{хорошим,} если в~нем можно переставить одну
цифру на~другое место и~получить девятизначное число, в~котором цифры
идут строго по~убыванию.
Сколько всего хороших чисел?

\item
Сколько существует десятизначных чисел, в~которых цифры не~возрастают?

\item
Назовем натуральное число \emph{интересным,} если оно~--- степень тройки или
представимо в~виде суммы различных степеней тройки.
Интересные числа занумеровали по~возрастанию.
Найдите сотое интересное число.

\end{problems}

