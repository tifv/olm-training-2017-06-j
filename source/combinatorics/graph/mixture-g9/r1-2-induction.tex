% $date: 2017-06-13
% $timetable:
%   g9r1:
%     2017-06-13:
%       1:

\worksheet*{Графы. Индукция}

% $authors:
% - Аскар Флоридович Назмутдинов

\begin{problems}

\item
Пусть $a_{1}$, \ldots, $a_{n}$~--- набор натуральных чисел, сумма которых равна
$2 n - 2$.
Докажите, что существует граф с~$n$~вершинами, степени которых равны
$a_{1}$, \ldots, $a_{n}$.

\item
Докажите, что существует граф с~$2 n$ вершинами, степени которых равны
$1$, $1$, $2$, $2$, \ldots, $n$, $n$.

\item
Правительство одной страны разработало 2 типа чипов, которое хочет вживить
каждому человеку.
Чипы первого типа заставляют людей всегда говорить правду, а~чипы второго типа
заставляют всегда лгать.
Можно~ли сделать так, чтобы после того как чипы вживят, каждый человек мог
сказать:
<<У~меня есть друг с~чипом второго типа>>?
(Все люди будут знать, у~кого какой чип.)

\item
В~стране каждые два города соединены дорогой с~односторонним движением.
\\
\subproblem
Докажите, что существует маршрут, проходящий по~всем городам по~одному разу.
\\
\subproblem
Оказалось, что из~любого города, проехав по~дорогам, можно попасть в~любой
другой.
Докажите, что можно проехать по~городам, побывав в~каждом по~разу и~вернувшись
в~начальный город.

\item
В~компании из~$n$ человек среди любых четверых есть человек, знакомый
с~остальными тремя.
Докажите, что есть человек, который знает всех остальных.

\item
В~вершинах некоторого графа с~$n$ вершинами записано по~два положительных
числа: синее и~красное, причем сумма синих равна сумме красных.
За~один ход можно изменить два синих числа в~концах любого одного
ребра так, чтобы чтобы они остались положительными и~их сумма сохранилась.
Докажите, что не~более чем за~$n-1$ ход можно добиться, чтобы в~каждой вершине
синее число стало равно красному.

\end{problems}

