% $date: 2017-06-07
% $timetable:
%   g9r3:
%     2017-06-07:
%       1:

\worksheet*{Графы}

% $authors:
% - Аскар Флоридович Назмутдинов

\begin{problems}

\item
В~некоторой компании у~каждого человека есть ровно двое друзей.
Докажите, что если каждый возьмется за~руки со~своими друзьями, то~образуется
один или несколько хороводов.

\item
В~доме отдыха 1999 отдыхающих.
Некоторые из~них знакомы между собой, причем любые двое незнакомых имеют среди
отдыхающих общего знакомого.
Каково наименьшее возможное число пар знакомых отдыхающих?

\end{problems}


\subsection*{Остовные деревья}

\begin{problems}

\item
\subproblem
Докажите, что в~дереве есть вершина степени~1.
\\
\subproblem
В~дереве $n$~вершин, сколько в~нем ребер?

\item
Докажите, что в~любом связном графе можно удалить вершину вместе со~всеми
выходящими из~нее ребрами так, чтобы он остался связным.

\item
В~графе $k$~вершин и~$2 k - 1$ ребер.
Докажите, что из~него можно удалить какой-то цикл (т.\,е. удалить все ребра
этого цикла, оставив вершины) так, чтобы граф остался связным.

\end{problems}


\subsection*{Подвешивание}

\begin{problems}

\item
В~государстве 2001 город, некоторые (но~не~все!) пары городов соединены
беспосадочными авиалиниями, причем из~любого города можно долететь до~любого
другого (возможно, с~пересадками).
Докажите, что министерство авиации может открыть новую авиалинию между
какими-то двумя городами так, что после этого из~любого города в~любой другой
можно будет долететь, сделав менее 1400 пересадок.

\item
В~стране 90~городов.
Некоторые пары городов соединены дорогами, не~проходящими через другие города.
Из~каждого города выходит хотя~бы три дороги.
Докажите, что существует несамопересекающийся циклический маршрут, состоящий
не~более, чем из~10 городов.

\end{problems}


\subsection*{Антиграф}

\begin{problems}

\item
В~группе из~100 людей среди любых троих есть человек, знающих обоих других.
Докажите, что из~этой группы можно выбрать компанию из~50 человек, в~которой
все знакомы друг с~другом.

\item
На~вечеринку пришло 19~гостей, причем среди любых трех из~них есть двое
знакомых.
Докажите, что гости могут разбиться на~5 групп, в~каждой из~которых все попарно
знакомы.

\end{problems}

