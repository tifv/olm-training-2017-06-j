% $date: 2017-06-08
% $timetable:
%   g9r3:
%     2017-06-08:
%       2:

\worksheet*{Графы-2}

% $authors:
% - Аскар Флоридович Назмутдинов

\subsection*{Гамильтоновость}

Степень вершины~$v$ графа~$G$ обозначается $d_{G}(v)$.
Наименьшая из~степеней вершин графа~$G$ обозначается $\delta(G)$.

\emph{Простым путем} в~графе называется путь, не~имеющий самопересечений
по~вершинам (то~есть никакая вершина не~встречается в~нем дважды).
Аналогично, \emph{простым циклом} называется цикл, не~имеющий самопересечений
по~вершинам.

\emph{Длиной пути (цикла)} называется количество его ребер.

\begin{problems}

\item
Пусть $\delta(G) \geq 2$.
Докажите, что
\\
\subproblem
в~графе~$G$ есть простой путь длины хотя~бы $\delta(G)$;
\\
\subproblem
в~графе~$G$ есть простой цикл длины хотя~бы $\delta(G) + 1$.

\item
Пусть $n > 2$, $v_{1} \ldots v_{n}$~--- максимальный простой путь в~графе~$G$,
причем \( d_{G}(v_{1}) + d_{G}(v_{n}) \geq n \).
Докажите, что в~графе есть простой цикл длины~$n$.

\end{problems}

\emph{Гамильтоновым путем} называется путь, проходящий по~каждой вершине графа
ровно один раз.
\emph{Гамильтоновым циклом} называется цикл, проходящий по~каждой вершине графа
ровно один раз.

\begin{problems}

\item\claim{Теорема Оре}
В~графе~$n$ вершин.
\\
\subproblem
Сумма степеней любых двух несмежных вершин больше либо равна~$n - 1$.
Докажите, что в~этом графе есть гамильтонов путь.
\\
\subproblem
Сумма степеней любых двух несмежных вершин больше либо равна~$n$.
Докажите, что в~этом графе есть гамильтонов цикл.

Выведите из~этого теорему Дирака:

\resetsubproblem
\claim{Теорема Дирака}
Пусть в~графе $N$ вершин.
Тогда
\\
\subproblem
если степени всех вершин не~меньше~$\frac{N - 1}{2}$, то~в~данном графе
существует гамильтонов путь;
\\
\subproblem
если степени всех вершин не~меньше~$\frac{N}{2}$, то~в~данном графе существует
гамильтонов цикл.

\item
Каждый из~$4034$ человек знаком не~менее чем с~$k$ из~остальных.
При каком наименьшем $k$ их гарантированно можно поселить в~двухместные номера
гостиницы так, чтобы каждый был поселен со~своим знакомым?

\end{problems}


\subsection*{Разное}

\begin{problems}

\item
\subproblem
В~кружке 53~ученика.
Известно, что если трое кружковцев попарно незнакомы друг с~другом, то~какие-то
двое из~них имеют в~кружке общего знакомого.
Докажите, что кто-то из~учеников имеет в~кружке хотя~бы 6~знакомых
\\
\subproblem
То~же самое для 49 учеников.

\item
В~группе из~100 человек среди любых 50~человек есть человек, который знает
всех из~этих 50.
Докажите, что найдутся 50 попарно знакомых людей.

\end{problems}

