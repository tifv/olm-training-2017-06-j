% $date: 2017-06-02
% $timetable:
%   g8r3:
%     2017-06-03:
%       1:

\worksheet*{Графы. Пути и~расстояния}

% $authors:
% - Андрей Юрьевич Кушнир

\definition
\emph{Расстоянием} $\rho(u, v)$ между двумя вершинами $u$ и~$v$ одной
компоненты связности графа $G$ назовём минимальное возможное число рёбер
в~маршруте из~$u$ в~$v$.
\emph{Диаметром} $d(G)$ связного графа $G$ назовём максимальное расстояние
между его вершинами.
%\emph{Шар} с~центром в~вершине $u$ и~радиусом $r$ определим как множество
%вершин графа, удалённых от~$u$ на~расстояние не~более чем~$r$.
%\emph{Радиус} $R(G)$ связного графа $G$ есть наименьший радиус шара, которым
%можно покрыть весь граф.

\begin{problems}

%\item
%Докажите \emph{неравенство треугольника}:
%$\rho(u, w) \leq \rho(u, v) + \rho(v, w)$ для любых вершин $u, v, w$.

%\item
%Докажите, что в~любом связном графе $G$ выполнено $R(G) \leq d(G) \leq 2R(G)$.
%Приведите примеры графов, когда эти неравенства обращаются в~равенства.

\item
Каждая вершина графа имеет степень не~менее $k$, где $k > 1$.
Докажите, что в~графе найдётся простой цикл длины не~менее $k + 1$.

\item
Диаметр связного графа $G$ больше $2$.
Докажите, что диаметр антиграфа к~$G$ не~больше $3$.

\item
В~графе между любыми двумя вершинами существует простая цепь чётной длины.
Докажите, что между любыми двумя вершинами существует простая цепь нечётной
длины.

\item
В~некотором графе $100$ вершин, $200$ рёбер и~нет простых циклов длины $3$ и~$4$.
Докажите, что в~графе найдутся два простых цикла, не~имеющих общих вершин.

\item
В~графе $100$ вершин;
степень каждой вершины не~меньше $3$.
Докажите, что в~графе существует простой цикл длины не~более $12$.

\item
Несколько деревень соединены дорогами, причём длина каждой дороги меньше $10$
км.
Известно, что из~любой деревни до~любой другой можно добраться, проехав меньше
$10$ км.
Одну дорогу закрыли, но~всё ещё можно добраться из~любой деревни до~любой
другой.
Докажите, что это можно сделать, проехав меньше $30$ км.
% (Дороги могут быть извилистыми, т.\,е. неравенство треугольника
% не~обязательно выполнено).

\end{problems}

