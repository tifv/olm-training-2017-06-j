% $date: 2017-06-11
% $timetable:
%   g8r1:
%     2017-06-11:
%       1:
%   g8r2:
%     2017-06-11:
%       2:

\worksheet*{Неявные графы}

% $authors:
% - Андрей Юрьевич Кушнир

\begin{problems}

\item
Куб $n \times n \times n$ разбит на~кубики $1 \times 1 \times 1$.
Какое минимальное количество граней $1 \times 1$ необходимо в~нем убрать, чтобы
из~любой его части можно было пробраться наружу?

\item
На~какое наименьшее число частей надо разрезать проволоку длиной
\\
\subproblemy{a} \rlap{$12$}\hphantom{трехмерного}
\qquad
\subproblemy{b} $32$,
\\
чтобы из~них можно было сложить каркас
\\
\subproblemy{a} трехмерного
\qquad
\subproblemy{b} четырехмерного
\\
кубика с~ребром $1$?
\begin{center}
    \jeolmfigure{../cube}
\qquad
    \jeolmfigure{../cube4}
\end{center}

%\item
%Клетки клетчатой плоскости раскрашены в~$10$ цветов, причем все цвета
%присутствуют.
%Два цвета назовем \emph{соседними}, если существуют две соседние по~стороне
%клетки, покрашенные именно в~эти два цвета
%(цвет может быть соседним с~самим собой).
%Какое минимальное число пар соседних цветов может быть?

\item
В~каждой клетке доски $m \times n$ прорезали одну из~диагоналей.
На~какое минимальное число частей она могла распасться?

\item
В~городе Прямоугольницке схема улиц представляет из~себя прямоугольную сетку
$m \times n$.
На~некоторых улицах (но~не~на~перекрестках) стоят полицейские и~записывают
направление и~момент времени проезжающих мимо машин.
Какое минимальное число полицейских должно стоять на~улицах города, чтобы можно
было однозначно восстановить любой замкнутый маршрут и~направление движения
автомобиля (машина не~ездит по~одной улице дважды)?

\item
Дан прямоугольник.
Провели $n - 1$ горизонтальных разрезов и~$n - 1$ вертикальных, изначальный
прямоугольник разрезался на~$n^2$ прямоугольников.
На~каждый из~этих $n^2$~прямоугольников положили карточку с~написанной на~ней
площадью этого маленького прямоугольника числом вниз.
Какое минимальное число карточек нужно перевернуть, чтобы узнать площадь
изначального прямоугольника?

%\item
%Дана прямоугольная таблица $n \times n$, состоящая из~прямоугольников разных
%размеров.
%За~один вопрос разрешается узнать площадь любого отдельного прямоугольника
%разбиения.
%За~какое наименьшее число вопросов удастся узнать площадь всего многоугольника?

\item
На~плоскости нарисовано $n$~кругов, причем любые два круга не~пересекаются,
но~могут касаться.
Каково максимальное количество точек касания?

\item
Поверхность кубика $5 \times 5 \times 5$ покрашена в~три цвета (в~черный и~еще
в~два) \emph{правильным образом}, т.\,е. любые два квадратика, соседние
по~стороне, раскрашены в~разные цвета.
Какое минимальное число квадратиков могло быть покрашено в~черный цвет?

\item
Петя поставил на~доску $50 \times 50$ несколько фишек, в~каждую клетку~---
не~больше одной.
Докажите, что Вася может поставить на~свободные поля этой~же доски не~более
99 новых фишек (возможно, ни~одной) так, чтобы по-прежнему в~каждой клетке
стояло не~больше одной фишки, и~в~каждой строке и~каждом столбце этой доски
оказалось четное количество фишек.

%\item
%Муравей ползает по~поверхности кубика $7 \times 7 \times 7$ вдоль диагоналей
%квадратиков $1 \times 1$ (поворачивать в~центре клетки нельзя).
%Могло~ли так оказаться, что он побывал в~каждом квадратике ровно один раз?

\end{problems}

