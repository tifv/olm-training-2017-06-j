% $date: 2017-06-11
% $timetable:
%   g9r1:
%     2017-06-11:
%       2:

\worksheet*{Зацикливание}

% $authors:
% - Аскар Флоридович Назмутдинов

\begin{problems}

\item
Школьник решил сдать Владимиру Алексеевичу зачет по~спецкурсу.
В.\,А. оставил на~дверях всех комнат школы записки вида
<<Я в~кабинете №\ldots>> и~исчез в~неизвестном направлении.
(Разные записки могут сообщать разную информацию.)
Школьник начал поиски В.\,А. с~5-го кабинета, руководствуясь этими указаниями.
\\
\subproblem
Докажите, что с~некоторого момента он начнет двигаться по~циклу.
\\
\subproblem
Можно~ли утверждать, что школьник когда-нибудь вернется в~5-й кабинет?
Верно~ли это утверждение, если все записки различны?

\item
Докажите, что если натуральное число $n$ не~делится ни~на~$2$, ни~на~$5$,
то~десятичное разложение дроби $1/n$ не~имеет предпериода
(то~есть зацикливание начинается с~первого~же знака после запятой).

\item
Некоторая комбинация поворотов граней вывела кубик Рубика из~собранного
положения.
Докажите, что кубик можно снова собрать, повторив эту~же комбинацию еще
несколько раз.

\item
Докажите, что для любого $n$ в~ряде Фибоначчи существует бесконечно много членов
\\
\subproblem имеющих остаток~$1$ при делении на~$n$;
\\
\subproblem делящихся на~$n$.

\item
На~бесконечной в~обе стороны ленте записан текст на~русском языке.
Известно, что в~этом тексте число различных кусков из~15 символов равно числу
различных кусков из~16 символов.
Докажите, что на~ленте записан <<периодический>> текст, например:
\\
\text{\ldots мамамыларамумамамылараму\ldots}

\item
Назовем сочетанием цифр несколько цифр, записанных подряд.
В~стране Роботландии некоторые сочетания цифр объявлены запрещенными.
Известно, что запрещенных сочетаний конечное число и~существует бесконечная
десятичная дробь, не~содержащая запрещенных сочетаний.
Докажите, что существует бесконечная периодическая десятичная дробь,
не~содержащая запрещенных сочетаний.

\item
В~тридесятом царстве ни~одна из~дорог не~заканчивается тупиком.
Рыцарь, Любящий Постоянство, выезжает из~своего замка и, доезжая до~любого
перекрестка, едет по~самой левой дороге.
Докажите, что в~конце концов он попадет таким образом обратно в~свой замок.

\item
По~кругу стоит несколько коробочек.
Каждая из~них может быть пустой или содержать один или несколько шариков.
Сначала из~какой-то коробочки берутся все шарики и~раскладываются по~одному
по~часовой стрелке, начиная со~следующей коробочки.
На~следующем ходу раскладывают шарики из~той коробочки, в~которую попал
последний шарик на~предыдущем ходу, и~т.\,д.
Докажите, что в~какой-то момент повторится начальное расположение шариков.

\item
На~проволоку, имеющую форму окружности, насажено несколько стальных шариков
одинакового размера.
В~некоторый момент шарики начинают двигаться с~одинаковыми скоростями,
но~некоторые~--- по~часовой стрелке, а~некоторые~--- против.
Сталкиваясь, шарики разлетаются с~теми~же скоростями в~противоположные стороны.
Докажите, что через некоторое время расположение шариков на~окружности совпадет
с~исходным, если:
\\
\subproblem шарики неразличимы;
\\
\subproblem все шарики различны.

\end{problems}

