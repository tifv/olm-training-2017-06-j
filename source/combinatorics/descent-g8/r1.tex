% $date: 2017-06-05
% $timetable:
%   g8r1:
%     2017-06-05:
%       3:
%   g8r2:
%     2017-06-06:
%       2:

\worksheet*{Индукция, спуск и~минимальный контрпример}

% $authors:
% - Андрей Юрьевич Кушнир

\begin{problems}

\item
Выпуклый $n$-угольник ($n \geq 3$) разбит непересекающимися диагоналями
на~треугольники (диагонали могут иметь общие концы).
Докажите, что найдутся хотя~бы две вершины, из~которых не~выходит ни~одной
диагонали.

\item
На~плоскости дано $n \geqslant 3$ точек, любые две из~которых соединены ровно
одной стрелкой.
\\
\subproblem
Докажите, что можно выбрать некоторую точку и, двигаясь по~стрелкам, обойти все
точки ровно по~одному разу.
\\
\subproblem
Известно, что из~любой точки можно добраться до~любой другой, двигаясь
по~стрелкам.
Докажите, что существует цикл, проходящий по~всем вершинам ровно по~одному
разу.

\item
Дана таблица $m \times n$.
Какое наибольшее число клеток в~ней можно отметить так, чтобы никакие три
центра отмеченных клеток не~образовывали прямоугольный треугольник?

\item
Клетки шахматной доски $100 \times 100$ раскрашены в~$4$ цвета таким образом,
что в~любом квадрате $2 \times 2$ все клетки разного цвета.
Докажите, что угловые клетки раскрашены в~разные цвета.

\item
В~каждой из~$n$ вершин связного графа лежат монеты, причем общее число монет
кратно $n$.
За~одну операцию разрешается переложить из~любой вершины несколько монет
в~соседнюю вершину.
Докажите, что не~более чем за~$n - 1$ операций можно добиться того, чтобы
во~всех вершинах графа стало поровну монет.

\item
Даны $n$~точек, некоторые из~которых соединены стрелками, причем любые две
точки соединены не~более чем одной стрелкой.
Известно, что из~любой точки можно добраться до~любой другой, двигаясь
по~стрелкам.
Докажите, что можно выкинуть несколько стрелок, оставив не~более $2n - 3$, так,
чтобы по-прежнему из~любой точки можно было добраться до~любой другой.

\item
\emph{Письменная задача.}
В~углу изначально белой доски $n \times n$ стоит ладья.
За~один ход разрешается непрерывно передвинуть ладью по~горизонтали или
по~вертикали, причем как только ладья покидает клетку, она окрашивается
в~черный цвет (<<ладья оставляет черный шлейф>>).
Ладье запрещено передвигаться через черные клетки, в~том числе останавливаться
на~них.
Два игрока по~очереди делают ходы, проигрывает тот, кто не~может сделать ход.
Кто выигрыет при правильной игре: начинающий, или его соперник?

%\item
%Петя и~Вася играют в~игру на~клетчатой доске $n \times n$.
%Изначально вся доска белая, за~исключением угловой клетки~--- она черная,
%и~в~ней стоит ладья.
%Игроки ходят по~очереди.
%Каждым ходом игрок передвигает ладью по~горизонтали или вертикали, при этом все
%клетки, через которые ладья перемещается (включая ту, в~которую она попадает),
%перекрашиваются в~черный цвет.
%Ладья не~должна передвигаться через черные клетки или останавливаться на~них.
%Проигрывает тот, кто не~может сделать ход;
%первым ходит Петя.
%Кто выиграет при правильной игре?
%% Всерос-2014, 11.2

\end{problems}

