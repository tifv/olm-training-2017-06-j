% $date: 2017-06-05
% $timetable:
%   g8r3:
%     2017-06-05:
%       3:

\worksheet*{Индукция, спуск и~минимальный контрпример}

% $authors:
% - Андрей Юрьевич Кушнир

\begin{problems}

\item
Выпуклый $n$-угольник ($n \geq 3$) разбит непересекающимися диагоналями
на~треугольники (диагонали могут иметь общие концы).
\\
\subproblem
Сколько диагоналей проведено?
\\
\subproblem
Докажите, что найдутся хотя~бы две вершины, из~которых не~выходит ни~одной
диагонали.

\item
В~каждой клетке доски $3 \times n$ стоит фишка одного из~трех цветов, причем
всего фишек каждого цвета на~доске поровну.
Внутри каждой из~трех строк разрешается переставлять фишки в~любом порядке.
Докажите, что их можно расставить так, что в~каждом столбце будут фишки всех
трех цветов.

\item
На~плоскости дано $n \geqslant 3$ точек, любые две из~которых соединены ровно
одной стрелкой.
\\
\subproblem
Докажите, что можно выбрать некоторую точку и, двигаясь по~стрелкам, обойти все
точки ровно по~одному разу.
\\
\subproblem
Известно, что из~любой точки можно добраться до~любой другой, двигаясь
по~стрелкам.
Докажите, что существует цикл, проходящий по~всем вершинам ровно по~одному
разу.

\item
Дана таблица $m \times n$.
Какое наибольшее число клеток в~ней можно отметить так, чтобы никакие три
центра отмеченных клеток не~образовывали прямоугольный треугольник?

\item
Клетки шахматной доски $100 \times 100$ раскрашены в~$4$ цвета таким образом,
что в~любом квадрате $2 \times 2$ все клетки разного цвета.
Докажите, что угловые клетки раскрашены в~разные цвета.

\item
В~каждой из~$n$ вершин связного графа записаны два целых неотрицательных числа:
красное и~синее;
известно, что сумма всех красных чисел равна сумме всех синих.
Изначально в~каждую вершину графа кладут монеты, количество которых совпадает
с~красным числом.
За~одну операцию разрешается переложить из~любой вершины несколько монет
в~соседнюю вершину.
Докажите, что не~более чем за~$n - 1$ операций можно добиться того, чтобы
количества монет в~вершинах совпали с~синими числами.

%\item
%В~стране $40$~городов, некоторые из~которых соединены ребрами.
%Известно, что из~любого города можно добраться до~любого другого.
%Докажите, что существует маршрут длины~$76$, проходящий через все города.
%\emph{Маршрут}~--- последовательность не~обязательно различных вершин, соседние
%элементы которой соединены ребром (ребра тоже могут повторяться).

\item
\emph{Письменная задача.}
Петя и~Вася играют в~игру на~клетчатой доске $n \times n$.
Изначально вся доска белая, за~исключением угловой клетки~--- она черная,
и~в~ней стоит ладья.
Игроки ходят по~очереди.
Каждым ходом игрок передвигает ладью по~горизонтали или вертикали, при этом все
клетки, через которые ладья перемещается (включая ту, в~которую она попадает),
перекрашиваются в~черный цвет.
Ладья не~должна передвигаться через черные клетки или останавливаться на~них.
Проигрывает тот, кто не~может сделать ход;
первым ходит Петя.
Кто выиграет при правильной игре?
% Всерос-2014, 11.2

\end{problems}

