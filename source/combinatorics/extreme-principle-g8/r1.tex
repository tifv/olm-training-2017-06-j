% $date: 2017-06-03
% $timetable:
%   g8r1:
%     2017-06-02:
%       3:
%   g8r2:
%     2017-06-02:
%       1:

\worksheet*{Принцип крайнего}

% $authors:
% - Андрей Юрьевич Кушнир

\begin{problems}

\item
Можно~ли на~клетчатой плоскости расставить конечное число ладей таким образом,
чтобы каждая из~них била не~менее трех других?

\item
В~каждой клетке
\\
\subproblem таблицы $100 \times 100$
\qquad
\subproblem бесконечной клетчатой плоскости
\\
расставлены числа, причем число в~любой клетке равно среднему арифметическому
чисел во~всех соседних по~стороне клетках.
Верно~ли, что все числа обязательно равны друг другу?

\item
Банда из~$25$ гангстеров устроила перестрелку, в~которой каждый из~них получил
пулю от~ближайшего гангстера (они сидели в~укрытиях и~не~перемещались, все
попарные расстояния между ними были различны).
Докажите, что какой-то гангстер не~подстрелил никого.

%\item
%Из~точки~$O$ плоскости проведено несколько лучей, так что угол между любыми
%двумя из~них меньше $120^{\circ}$.
%Докажите, что все лучи можно накрыть углом $120^{\circ}$ с~вершиной~$O$.

\item
В~вершинах $n$-угольника расставлены $m$~фишек ($m > n > 2$).
За~ход можно выбрать какую-нибудь вершину, снять с~нее две фишки (если они там
есть) и~добавить по~одной фишке к~ее соседним вершинам.
Докажите, что если через несколько ходов в~каждой вершине будет находиться
столько~же фишек, сколько в~ней было изначально, то~количество совершённых
ходов будет кратно $n$.

\item
На~столе лежат (необязательно одинаковые) монеты без наложений.
Докажите, что одну из~них можно выдвинуть, не~задевая остальных.

\item
Докажите, что у~выпуклого многогранника найдутся две грани с~одинаковым числом
сторон.

\item
На~плоскости проведено несколько (больше одной) прямых \emph{общего положения},
т.\,е. попарно непараллельных прямых, никакие три из~которых не~пересекаются
в~одной точке.
Эти прямые разбивают плоскость на~части.
Докажите, что среди частей разбиения есть хотя~бы
\\
\subproblem одна часть
\qquad
\subproblem три части
\\
являющиеся внутренностями угла.

\item
\begin{minipage}[t][][t]{0.79\linewidth}
На~клетчатой плоскости нарисован выпуклый пятиугольник, все вершины которого
находятся в~узлах решетки.
\\
\subproblem
Докажите, что строго внутри пятиугольника есть по~крайней мере один узел
решетки.
\\
\subproblem
Докажите, что внутри или на~границе пятиугольника, образованного диагоналями
исходного, есть по~крайней мере один узел решетки.
\end{minipage}\hfill
\begin{minipage}[t][][b]{0.18\linewidth}
    \vspace{-1ex}
    \jeolmfigure[width=\linewidth]{../pentagon}
\end{minipage}

\item \emph{Теорема Сильвестра.}
\subproblem
На~плоскости дано конечное множество точек.
Известно, что любая прямая, соединяющая какие-то две точки множества, содержит
по~крайней мере еще одну точку множества.
Докажите, что все точки множества лежат на~одной прямой.
\\
\subproblem
На~плоскости дано конечное множество непараллельных друг другу прямых.
Известно, что через любую точку пересечения прямых множества проходит
по~крайней мере еще одна прямая множества.
Докажите, что все прямые множества пересекаются в~одной точке.

\end{problems}

