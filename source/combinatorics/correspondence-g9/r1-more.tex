% $date: 2017-06-06
% $timetable:
%   g9r1:
%     2017-06-06:
%       1:

\worksheet*{Соответствия. Добавка}

% $authors:
% - Аскар Флоридович Назмутдинов

\begin{problems}

\item
Петя загадал произвольное 23-значное число.
Вася утверждает, что может посчитать количество 2017-значных чисел, из~которых 
Петя может получить свое число зачеркиванием нескольких цифр.
(Вася не~знает Петино число.)
Прав~ли Вася?

\item
По~кругу расставили 2017 чисел.
Проведем стрелки между числами так, чтобы из~каждого числа выходило ровно
по~1 стрелке, и~в~каждое число входило ровно по~1 стрелке.
Докажите, что количество способов расставить стрелки так, чтобы числа разбились
на~23 цикла, равно количеству способов расставить эти числа в~ряд так, чтобы
было ровно 23~числа, которые больше всех чисел, стоящих перед ним.

\item
Обозначим за~$A(n)$ число последовательностей $a_{1}$, $a_{2}$, \ldots,
$a_{k}$, для которых
\(
    a_{1} \geq a_{2} \geq \ldots \geq a_{k} > 0
\) и~\(
    a_{1} + a_{2} + \ldots + a_{k} = n
\), причем каждое число $a_{i} + 1$ является степенью двойки.
Обозначим за~$B(n)$ число последовательностей $b_{1}, \ldots, b_{m}$, для которых
\(
    b_{1} \geq b_{2} \geq \ldots \geq b_{m} > 0
\) и~\(
    b_{1} + \ldots + b_{m} = n
\), причем $b_{j} \geq 2 b_{j+1}$ при всех $j$ от~$1$ до~$m - 1$.
Докажите, что $A(n) = B(n)$ для всех натуральных $n$.

\end{problems}

