% $date: 2017-06-04
% $timetable:
%   g9r3:
%     2017-06-04:
%       2:

\worksheet*{Комбинаторика-2}

% $authors:
% - Владислав Викторович Новиков

\begin{problems}

\item
Найдите количество решений уравнения в целых неотрицательных числах:
\\
\subproblem $x + y + z + t = 1000$;
\\
\subproblem $x + y + z + t \leq 1000$;
\\
\subproblem $x + y + z = 1000$, при этом $x, y, z \leq 750$;

\item
\subproblem
В ряд
\qquad
\subproblem
По кругу
\\
стоят $n$~предметов.
Сколько способов выбрать $k$ из них так, чтобы никакие два не стояли рядом?

\item
Сколько существует разных способов разбить число 2014 на~натуральные слагаемые,
которые приблизительно равны?
Слагаемых может быть одно или несколько.
Числа называются \emph{приблизительно равными,} если их разность не~больше 1.
Способы, отличающиеся только порядком слагаемых, считаются одинаковыми.

\item
Для каждого трехзначного числа берем произведение его цифр, а~затем эти
произведения, вычисленные для всех трехзначных чисел, складываем.
Сколько получится?
(Разумеется, если хотя~бы одна из~цифр числа~--- ноль, то~и~произведение равно
нулю).

\item
\subproblem
Докажите, что количество разбиений числа~$n$ в~сумму не~более чем
$k$~слагаемых равно количеству разбиений числа~$n$ в~сумму слагаемых,
не~превосходящих $k$.
\\
\subproblem
Докажите, что количество разбиений числа~$n$ равно количеству разбиений
числа~$2n$ в~сумму ровно $n$~слагаемых.

\item
Десятизначное число назовем \emph{хорошим,} если у~него сумма первой
и~последней цифр равна сумме второй и~предпоследней, равна сумме третьих
с~конца и~с~начала, и~т.\,д.
Каких хороших чисел больше: четных, или нечетных?

\item
Докажите, что у~точного квадрата делителей вида $4k + 1$ больше, чем делителей
вида $4 k - 1$.

\end{problems}

