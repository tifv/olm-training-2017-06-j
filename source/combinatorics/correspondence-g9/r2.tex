% $date: 2017-06-04
% $timetable:
%   g9r2:
%     2017-06-04:
%       1:

\worksheet*{Соответствия}

% $authors:
% - Владислав Викторович Новиков

\begin{problems}

\item
Сколько существует разных способов разбить число 2014 на~натуральные слагаемые,
которые приблизительно равны?
Слагаемых может быть одно или несколько.
Числа называются \emph{приблизительно равными,} если их разность не~больше 1.
Способы, отличающиеся только порядком слагаемых, считаются одинаковыми.

\item
Для каждого трехзначного числа берем произведение его цифр, а~затем эти
произведения, вычисленные для всех трехзначных чисел, складываем.
Сколько получится?
(Разумеется, если хотя~бы одна из~цифр числа~--- ноль, то~и~произведение равно
нулю).

\item
\subproblem
Докажите, что количество разбиений числа~$n$ в~сумму не~более чем
$k$~слагаемых равно количеству разбиений числа~$n$ в~сумму слагаемых,
не~превосходящих $k$.
\\
\subproblem
Докажите, что количество разбиений числа~$n$ равно количеству разбиений
числа~$2n$ в~сумму ровно $n$~слагаемых.

\item
Докажите, что число всех цифр в~последовательности
$1$, $2$, $3$, \ldots, $10^{k}$ равно числу всех нулей в~последовательности
$1$, $2$, $3$, \ldots, $10^{k+1}$.

\item
Десятизначное число назовем \emph{хорошим,} если у~него сумма первой
и~последней цифр равна сумме второй и~предпоследней, равна сумме третьих
с~конца и~с~начала, и~т.\,д.
Каких хороших чисел больше: четных, или нечетных?

\item
Докажите, что у~точного квадрата делителей вида $4k + 1$ больше, чем делителей
вида $4 k - 1$.

\item
В~школе учатся 400 человек, из~них 200 двоечников и~200 отличников.
На~Новый Год Дед Мороз привез мешок, в~котором есть 800 конфет
<<Миндаль Иванович>>.
Он хочет раздать их все детям, причем каждый двоечник должен получить не~больше
одной конфеты, а~каждый отличник~--- хотя~бы 2 конфеты, причем четное
количество.
Директор школы решил, кроме того, поощрить отличников, приготовив
600 мандаринов, которые хочет раздать так, чтобы каждому отличнику досталось
не~менее одного.
У~кого больше способов раздать свои подарки?


\end{problems}

