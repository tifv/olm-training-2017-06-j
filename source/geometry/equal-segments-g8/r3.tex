% $date: 2017-06-04
% $timetable:
%   g8r3:
%     2017-06-04:
%       3:

\worksheet*{Перекладывание отрезков}

% $authors:
% - Алексей Вадимович Доледенок

\begin{problems}

\item
В~трапеции $ABCD$ с~основаниями $AD$ и~$BC$ выполнено $\angle ABD = 90^{\circ}$
и~$AD = BC + CD$.
Найдите отношение $AD$ к~$BC$.

\item
В~треугольнике $ABC$ биссектриса~$AE$ равна по~длине отрезку~$CE$.
Также известно, что $2 AB = AC$.
Найдите величину угла $B$.

\item
В~равнобедренном треугольнике $ABC$ ($AB = BC$) на~боковую сторону~$BC$ опущена
высота~$AH$.
Точка~$L$~--- основание перпендикуляра из~$H$ на~сторону~$AB$.
Оказалось, что $AL = AB / 4$.
Найдите угол $ABC$.

\item
В~выпуклом четырехугольнике $ABCD$ выполнено $\angle B = \angle C = 120^{circ}$
и~$BC + CD = AB$.
Докажите, что $AC = AD$.

\item
Пусть $AL$~--- биссектриса треугольника $ABC$.
Известно, что $\angle ABC = 2 \angle ACB + \angle BAC$.
Докажите, что $AB + CL = AC$.

\item
Биссектрисы углов треугольника $ABC$ пересекаются в~точке~$I$.
Известно, что $CA + AI = BC$.
Найдите отношение углов $BAC$ и~$CBA$.

\item
В~треугольнике $ABC$ угол~$B$ вдвое больше угла~$C$, а~угол~$A$~--- тупой.
Точка~$K$ на~стороне~$BC$ такова, что угол $KAC$~--- прямой.
Докажите, что $KC = 2 AB$.

\item
На~гипотенузе~$AC$ прямоугольного треугольника $ABC$ выбрана такая точка~$D$,
что $BC = CD$.
На~катете~$BC$ взята такая точка~$E$, что $DE = CE$.
Докажите равенство $AD + BE = DE$.

\item
Диагональ~$AC$ выпуклого четырехугольника $ABCD$ делится точкой пересечения
диагоналей пополам.
Известно, что $\angle ADB = 2 \angle CBD$.
На~диагонали~$BD$ нашлась такая точка~$K$, что $CK = KD + AD$.
Докажите, что $\angle BKC = 2 \angle ABD$.

%\item
%В~выпуклом четырехугольнике $ABCD$ выполняются равенства $\angle B = \angle C$
%и~$CD = 2 AB$.
%На~стороне~$BC$ выбрана точка~$X$ такая, что $\angle BAX = \angle CDA$.
%Докажите, что $AX = AD$.

%\item
%Дан прямоугольный треугольник $ABC$, в~котором $\angle A = 90^\circ$.
%$BL$~--- биссектриса угла~$B$, а~точка~$K$ на~стороне~$BC$ такова, что
%$\angle BLK = 90^\circ$.
%Оказалось, что $3 KC = 2 (BC - AB)$.`
%Найдите $\angle C$.

\end{problems}

