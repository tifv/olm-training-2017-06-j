% $date: 2017-06-02
% $timetable:
%   g8r3:
%     2017-06-02:
%       3:

\worksheet*{Центральная симметрия}

% $authors:
% - Алексей Вадимович Доледенок

\begin{problems}

\item
Докажите, что если в~треугольниках $ABC$ и~$A'B'C'$ равны стороны $BA = B'A'$
и~$BC = B'C'$ и~медианы $BM = B'M'$, то~эти треугольники равны.

\item
В~треугольнике $ABC$ провели медиану~$BM$.
Оказалось, что сумма углов $A$ и~$C$ равна углу $ABM$.
Найдите отношение медианы~$BM$ к~стороне~$BC$.

\item
Точка~$K$~--- середина гипотенузы~$AB$ прямоугольного треугольника $ABC$.
На~катетах $AC$ и~$BC$ выбраны точки $M$ и~$N$ соответственно так, что
угол $MKN$~--- прямой.
Докажите, что из~отрезков $AM$, $BN$ и~$MN$ можно составить прямоугольный
треугольник.

\item
В~треугольнике $ABC$ точка~$M$~--- середина~$AC$.
На~стороне~$BC$ взяли точку~$K$ так, что угол $BMK$ прямой.
Оказалось, что $BK = AB$.
Найдите $\angle MBC$, если $\angle ABC = 110^{\circ}$.

\item
На~медиане~$AM$ треугольника $ABC$ нашлась такая точка~$K$, что  $AK = BM$.
Кроме того, $\angle AMC = 60^{\circ}$.
Докажите, что $AC = BK$.

\item
В~треугольнике $ABC$ медиана, проведённая из~вершины~$A$ к~стороне~$BC$,
в~четыре раза меньше стороны~$AB$ и~образует с~ней угол $60^{\circ}$.
Найдите угол $BAC$.

\item
В~центре квадратного пруда плавает ученик.
Внезапно к~вершине квадрата подошёл учитель.
Учитель не~умеет плавать, но~бегает в~4~раза быстрее, чем ученик плавает.
Ученик бегает быстрее.
Сможет~ли он убежать?

\item
В~выпуклом четырехугольнике $ABCD$ стороны $AB$, $BC$ и~$CD$ равны,
$M$~--- середина стороны~$AD$.
Известно, что угол $BMC$ равен $90^\circ$.
Найдите угол между диагоналями четырехугольника $ABCD$.

\end{problems}

