% $date: 2017-06-10
% $timetable:
%   g9r1:
%     2017-06-10:
%       1:
%   g9r2:
%     2017-06-11:
%       2:

\worksheet*{Касательные к~окружности}

% $authors:
% - Фёдор Константинович Нилов

\begin{problems}

\item
Дан треугольник $ABC$.
\\
\subproblem
Вписанная окружность касается его стороны~$AC$ в~точке~$X$, а~вневписанная
окружность касается этой стороны в~точке~$Y$.
Докажите, что $AX = CY$.
\\
\subproblem
Вневписанные окружности касаются его сторон $AB$ и~$BC$ в~точках $K$ и~$L$
соответственно.
Докажите, что $AK = CL$.

\item
\subproblem
Дан описанный четырехугольник $ABCD$.
В~треугольники $ABC$ и~$ACD$ вписаны окружности.
Докажите, что они касаются диагонали $AC$ в~одной и~той~же точке.
\\
\subproblem
Дан произвольный четырехугольник $ABCD$.
В~треугольники $ABC$ и~$ACD$ вписаны окружности, которые касаются
диагонали~$AC$ в~точках $K$ и~$L$.
В~треугольники $ABD$ и~$BCD$ вписаны окружности, которые касаются
диагонали~$BD$ в~точках $M$ и~$N$.
Докажите, что $KL = MN$.

\item
Из~центра каждой из~двух данных окружностей проведены касательные к~другой
окружности.
Докажите, что хорды, соединяющие точки пересечения касательных с~окружностями, 
равны.

\item
Дан правильный треугольник $ABC$.
Через вершину~$B$ проводится проводится произвольная прямая~$l$, лежащая вне
треугольника.
Окружность~$\alpha$ касается прямой~$l$, стороны~$AB$ и~продолжения~$CA$
за~точку~$A$.
Окружность~$\beta$ касается прямой~$l$, стороны~$BC$ и~продолжения~$CA$
за~точку~$C$.
Докажите, что сумма радиусов окружностей $\alpha$ и~$\beta$ не~зависит
от~прямой~$l$.

\item
Дан вписанный четырехугольник $ABCD$.
В~треугольники $ABC$ и~$ACD$ вписаны окружности с~радиусами $r_1$ и~$r_2$.
В~треугольники $ABD$ и~$BCD$ вписаны окружности с~радиусами $r_3$ и~$r_4$.
Докажите, что
\\
\subproblem центры этих окружностей образуют прямоугольник;
\\
\subproblem $r_1 + r_2 = r_3 + r_4$.

\item
$D$~--- точка на~стороне~$BC$ треугольника $ABC$.
B треугольники $ABD$, $ACD$ вписаны окружности, и~к~ним проведена общая внешняя
касательная (отличная от~$BC$), пересекающая $AD$ в~точке~$K$.
Докажите, что длина отрезка~$AK$ не~зависит от~положения точки~$D$ на~$BC$.

\item
Внутри угла $AOD$ проведены лучи $OB$ и~$OC$, причем
$\angle AOB = \angle COD$.
В~углы $AOB$ и~$COD$ вписаны непересекающиеся окружности.
Докажите, что точка пересечения общих внутренних касательных к~этим окружностям
лежит на~биссектрисе угла $AOD$.

\item
Дан треугольник $ABC$.
Вневписанные окружности касаются сторон $AB$ и~$BC$ в~точках $X$ и~$Y$.
Точки $M$ и~$N$~--- середины отрезков $AC$ и~$XY$.
Докажите, что $MN$ параллельна биссектрисе угла~$B$.

\end{problems}

