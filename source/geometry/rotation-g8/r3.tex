% $date: 2017-06-03
% $timetable:
%   g8r3:
%     2017-06-03:
%       3:

\worksheet*{Крутой поворот}

% $authors:
% - Алексей Вадимович Доледенок

\begin{problems}

\item
В~ромбе $ABCD$ угол $ABC$ равен $120^{\circ}$.
На~сторонах $AB$ и~$BC$ взяты точки $P$ и~$Q$ так, что $AP = BQ$.
Докажите, что треугольник $PQD$ равносторонний.

\item
На~сторонах $AB$ и~$BC$ треугольника $ABC$ во~внешнюю сторону построены
правильные треугольники $ABK$ и~$BCL$.
Докажите, что отрезки $AL$ и~$KC$ имеют одинаковую длину и~найдите угол между
ними.

\item
Внутри квадрата $ABCD$ дана точка~$P$.
Докажите, что прямая, проходящая через точку~$A$ перпендикулярно $BP$, прямая,
проходящая через точку~$B$ перпендикулярно $CP$, прямая, проходящая через
точку~$C$ перпендикулярно $DP$, и~прямая, проходящая через точку~$D$
перпендикулярно $AP$, пересекаются в~одной точке.

\item
Дан треугольник $ABC$.
На~его сторонах $AB$ и~$BC$ построены внешним образом квадраты $ABMN$ и~$BCPQ$.
Докажите, что центры этих квадратов и~середины отрезков $MQ$ и~$AC$ образуют
квадрат.

\item
На~сторонах $AB$ и~$BC$ дельтоида $ABCD$ ($AB = AD$, $CB = CD$) построены
правильные треугольники $ABE$ (наружу) и~$BCF$ (внутрь).
Докажите, что точки $D$, $E$ и~$F$ лежат на~одной прямой.

\item
На~сторонах $AB$ и~$BC$ треугольника $ABC$ отмечены точки $P$ и~$Q$
соответственно.
Оказалось, что $AP + CQ = AC$.
Докажите, что точки $P$ и~$Q$ равноудалены от~точки пересечения биссектрис
треугольника $ABC$.

\item
Дан правильный шестиугольник $ABCDEF$.
Докажите, что точка~$A$ и~середины отрезков $BD$ и~$EF$ являются вершинами
правильного треугольника.

\item
Внутри треугольника $ABC$ нашлась точка~$T$ такая, что
\( \angle ATB = \angle BTC = \angle CTA = 120^{\circ} \).
Докажите, что для любой точки~$P$ верно неравенство
\( AP + BP + CP \geq AT + BT + CT \).

\item
На~двух сторонах $AB$ и~$BC$ правильного $2n$-угольника взяты соответственно
точки $K$ и~$N$, причем угол $KEN$, где $E$ ~--- вершина, противоположная $B$,
равен $\frac{180^{\circ}}{2n}$.
Докажите, что $NE$~--- биссектриса угла $KNC$.

%\item
%Угол~$B$ при вершине равнобедренного треугольника $ABC$ равен $120^{\circ}$.
%Из~вершины~$B$ выпустили внутрь треугольника два луча под углом $60^\circ$ друг
%к~другу, которые, отразившись от~основания~$AC$ в~точках $P$ и~$Q$, попали
%на~боковые стороны в~точки $M$ и~$N$.
%Докажите, что площадь треугольника $PBQ$ равна сумме площадей треугольников
%$AMP$ и~$CNQ$.

%\item
%Пусть $O$~--- центр описанной окружности треугольника $ABC$.
%На~сторонах $AB$ и~$BC$ выбраны точки $M$ и~$N$ соответственно, причем
%\( 2 \angle MON = \angle AOC \).
%Докажите, что периметр треугольника $MBN$ не~меньше стороны~$AC$.

\end{problems}

