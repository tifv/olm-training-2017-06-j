% $date: 2017-06-12
% $timetable:
%   g9r1:
%     2017-06-12:
%       2:
%   g9r2:
%     2017-06-13:
%       1:

\worksheet*{Геометрический винегрет}

% $authors:
% - Фёдор Константинович Нилов

\begin{problems}

\item
В~треугольнике $ABC$ биссектриса~$AK$ перпендикулярна медиане~$CL$.
Докажите, что в~треугольнике $BKL$ также одна из~биссектрис перпендикулярна
одной из~медиан.

\item
Из~листа бумаги в~клетку вырезали квадрат $2 \times 2$.
Используя только линейку без делений и~не~выходя за~пределы квадрата, разделите
его диагональ на~6 равных частей.

\item
Квадрат и~прямоугольник одинакового периметра имеют общий угол.
Докажите, что точка пересечения диагоналей прямоугольника лежит на~диагонали
квадрата.

\item
Дан прямоугольный треугольник $ABC$ с~гипотенузой~$AC$
и~углом $A = 50^{\circ}$.
На~катете~$CB$ взяли точки $X$ и~$Y$ такие, что
$\angle CAX = \angle BAY = 10^{\circ}$.
Докажите, что $CX : YB = 2 : 1$.

\item
В~остроугольном треугольнике один из~углов равен 60 градусов.
Докажите, что прямая, проходящая через центр описанной окружности и~точку
пересечения высот, отсекает от~него равносторонний треугольник.

\item
Дан треугольник $ABC$.
Обозначим через $A'$, $B'$ и~$C'$ основания внешних биссектрис данного
треугольника.
Докажите, что данные точки лежат на~одной прямой.

\item
Дан равнобедренный треугольник $ABC$ с~углом~$\alpha$ при вершине~$A$.
Дуга~$BC$ с~градусной мерой~$\beta$ построена вовсе треугольника.
Нашлись два луча, проходящие через вершину~$B$, которые делят сторону~$BC$
и~дугу~$BC$ на~три равные части.
Найдите отношение $\alpha$ и~$\beta$.

\end{problems}

